\documentclass{article}
\usepackage{graphicx} % Required for inserting images
\usepackage{amsmath} 
\usepackage{amsfonts} % moet erbij om verzamelingen N,R... netjes te krijgen

% mark paragraphs with empty line instead of indented first line
\setlength{\parindent}{0em}
\setlength{\parskip}{1em}

\title{Nota's hoorcolleges bewijzen en redeneren}
\author{Raf Meeusen}
\date{2023-2024}

\begin{document}

\maketitle

\section{Les 1, 27 sept}

Termen: \begin{itemize}
    \item proposities
    \item predicaten
\end{itemize}


voorbeeld predicaat: $x^2 \geq 0 $
(vrije variabele erin; waar of onwaar pas te bepalen als meer over x geweten is; bvb. reeel of complex) 

Een predicaat wordt een propositie als de vrije variabelen gespecifieerd zijn. 
Bvb.: voor alle reele getallen $x$ geldt dat $x^2 \geq 0$  ; dit is wel een propositie! 

Nog termen: 
\begin{itemize}
    \item beweringen
    \item connectief/ connectieven / logische connectieven (of , en, als ... dan etc.) 
    \item waarheidstabel (op te stellen voor elk connectief) 
\item implicaties (dit is de "als ... dan"; geen simpele) 
\item equivalenties ("als en slecht als" / asa / $\Leftrightarrow$) 

\end{itemize}


Implicaties: 
hoe begrijpen? als P waar is, dan \emph{moet} Q ook waar zijn; (dus als P niet waar is, kunnen we niks over Q zeggen) 
Voorbeeld: als $\pi < 3$, dan is 1+1=100 ; deze implicatie is "WAAR" !! 
Ook onthouden: niks te maken met causaal verband !! 
Voorbeeld: als 1+1=2, dan is $\pi$ irrationaal ; deze implicatie is "WAAR" !! 

Equivalentie: zelfde als twee implicaties die geldig zijn ($P=>Q$ en $Q=>P$). 
$P <=> Q $ betekent zelfde als : $(P=>Q)$  EN  $(Q=>P) $

Raadsel. 3 goden, allen oneindig slim. 1 god geeft systematisch juiste antwoord, 1 god geeft systematisch foute antwoord, en derde god geeft willekeurig antwoord. 
Godentaal: da / na ; maar we weten niet of da ja is of nee. 
We mogen 3 vragen stellen aan die goden, en bepalen welke god welke waarheid spreekt. 

"Soorten" bewijzen: 
\begin{itemize}
    \item rechtstreeks bewijs (cursus: direct bewijs) 
    \item gevalsonderscheid
\end{itemize}


Stappen in een bewijs: NIET met $=>$ aangeven. 
onderscheid: 
symbool $=>$ is een logisch connectief 
de stappen in een bewijs zijn een "verhaal". 
(analogie: zinsdeel , zin, verhaal) 

Achterwaarts construeren: 
eerder een techniek om een bewijs te vinden; niet zozeer een soort bewijs. 

Als je bewijs leest, denk je: hoe hebben ze dit ooit gevonden. Wel, misschien hebben ze eerst achterwaarts geconstrueerd, maar die redenering niet mee opgeschreven. Het opgeschreven bewijs is dan misschien een direct bewijs, maar het kan lijken op "hoe hebben ze dit gevonden"? "dit komt uit het niets". 

Nog soorten: 
\begin{itemize}
    \item Bewijs uit het ongerijmde 
\item Bewijs met contrapositie
\item Bewijs met volledige inductie
\end{itemize}

(volledige inductie: typisch voor uitspraken over de natuurlijke getallen)
Opmerking: natuurlijke getallen bevatten 0 (niet zo in alle landen!) 

Inductie stappen: basisstap, inductiestap (met inductiehypothese: "stelling is geldig voor n")

Stelling: formule voor n-de getal uit rij van Fibonacci: $a_n = 1/\sqrt{5} * (...) $, voor alle natuurlijke getallen n. 
Bewijs in cursus met volledige inductie. 

\section{Les 2, 4 okt}

Hoofdstuk 2. Verzamelingen en kwantoren. Betekenis element ($x \in X$). Bekende verzamelingen $\mathbb{N}, \mathbb{Z}, \mathbb{Q}, \mathbb{R}, \mathbb{C}, \mathbb{N}_0, ... $
Deelverzameling $A \subset B$. Onze afspraak: deelverzameling mag ook gelijke verzameling zijn (niet bij iedereen de conventie, er bestaat ook een symbool $\subseteq$, maar laat ons dat niet gebruiken). De lege verzameling $\emptyset$. Opsomming elementen $\{a,b,c\}$. Via beperkende eigenschap: $\{x\in \mathbb{R} \mid x>2 \} $. Of een formule: $\{x^5 -3x  \mid  x \in \mathbb{Z} \} $

Bewerkingen met verzamelingen: $A \cup B$, $A \cap B$. Universum U, en complement-verzameling $A^C = U \setminus A$. Distributiviteit van doorsnede op unie. 

Eigenschap: complement van een doorsnede is gelijk aan unie van complementen: $(A \cap B)^C = A^C \cup B^C$. Kan je aantonen met Venn-diagram, en de vier mogelijkheden beschouwen (in A en in B, in A en niet in B, ...). Maar kan ook anders. Typische mogelijkheid voor gelijkheid van twee verzamelingen $A en B$, is in twee stappen. Stap 1: bewijs dat $A \subset B$, en stap 2: bewijs dat $B \subset A$. 

Voorbeeldbewijs in les gezien van $(A \cap B)^C = A^C \cup B^C$. Principe voor eerste inclusie: neem willekeurige $x$ uit $(A \cap B)^C$. Dan zit  $x$ niet in $A \cap B$. En dus geldt: $(x\notin A)$ of $(x\notin B)$. Dus geldt: $(x \in A^C)$ of $(x\in B^C)$. En daarom ... En dan nog tweede inclusie bewijzen. En klaar. 

De machtsverzameling van verzameling $X$, genoteerd als $P(X)$, verzameling van alle deelverzamelingen. Noot: wij noteerden dat altijd $2^X$. Tricky: $P(\emptyset) = \{ \emptyset \}$, dus niet gewoon gelijk aan de lege verzameling. 

Cartesisch product van twee verzamelingen: $X \times Y$. 

Algemene unies en doorsnedes: unie of doorsnede gebaseerd op verzamelingen met een index. 

Kwantoren: "voor elke": $\forall$  en "er bestaat" $\exists$. Men noemt ze respectievelijk  de universele en de existenti\"ele kwantor. Dan in les even op andere manier geschreven (verzameling gelijk aan A voor de universele, en verzameling gelijk aan lege verzameling voor de existenti\"ele). Dat blijkt wel handig om bvb. volgende uitspraak als waar of onwaar te bekijken: $\forall x \in \emptyset : x>2$. Die is niet evident, en die is waar, want als je het herschrijft in die andere vorm, klopt het helemaal. 
Daarna ook nog andere uitdrukking met implicatie bekeken. Mmm, opnieuw implicaties, blijft lastig om over na te denken. Als linkerkant niet waar is, is de implicatie altijd waar. (maar daarom niet de rechterkant!)... 

Combinaties kwantoren. Volgorde is van belang. 

Negatie ($\lnot$) van de $\forall$ wordt een $\exists$. Negatie van de $\exists$ wordt een $\forall$. 


\section{Les 3, 11 okt}

Hoofdstuk 3. Relaties. Definitie als deelverzameling van cartesisch product. Dus een koppel $(x,y)$ kan element zijn van de relatie $R$. Iets formeler kan een relatie ook gedefinieerd worden als drietal $(R,X,Y)$. 

Inverse relatie $R^{-1}$: koppels $(y,x)$ met $(x,y) \in R$. 

Samenstelling van relaties R en S, notatie $S \circ R$. Engels: composition. Relaties \emph{op een verzameling} is de formulering als $X=Y$. Graaf, gerichte graaf: met pijlen de relatie tekenen. 

Equivalentierelaties. (andere: orde-relaties, zie later; nog andere: functies, zie volgende week). Het zijn relaties op X, die reflexief, symmetrisch en transitief zijn. Voorbeeld: "heeft dezelfde verjaardag als" in verzameling studenten. 

Definitie partitie $\mathbf{P}$ van X. Partitie is deelverzameling van de machtsverzameling $P(X)$. Via eigenschappen gedefinieerd. 

Stelling die partities en equivalentierelaties aan mekaar koppelt (1-op-1 relatie ertussen). Equivalentieklassen gedefinieerd tussendoor in de les. Notatie $[x]_R$. Notatie met tilde $\sim$ voor equivalentierelaties en ook bij equivalentieklasse gebaseerd op relatie.  Dan soort bewijs gezien, dat elke equivalentierelatie een partitie definieert, door elke van de 3 eigenschappen na te gaan. 

Het woord "lemma" gebruikt: voor hulp-resultaat, een minder belangrijke stelling. 

Even terug naar eigenschappen van relaties: reflexief, symmetrisch, transitief: opdracht, probeer eens alle combinaties te vinden: reflexief, niet symmetrisch, niet transitief, etc. 


\section{Les 4, 18 okt}


\section{KEEP ME TO JUMP TO END}

\end{document}