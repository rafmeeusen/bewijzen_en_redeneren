\documentclass[hidelinks,11pt,a4paper]{article}
\usepackage[dutch]{babel}
\usepackage{amssymb, amsthm, hyperref, amsmath} 
\newtheorem{stelling}{Stelling}
\newtheorem{lemma}[stelling]{Lemma}

\title{LaTeX opdracht Bewijzen en Redeneren 2023}
\author{Raf Meeusen, s0050110}
\date{11 november 2023}


\begin{document}
\maketitle

%%%%%%%%%%%%%%%%%%%%%%%%%%%%%%%%%%%%%%%%%%%%%%%%%%%%%%%%%%%%%%%%%%%%%%%%%%%%%%%%%%%%%%%%%%%%%%%%%%%%%%%%%%%%%
\section{Paul Cohen}
% Opdracht 1 Geef een korte beschrijving van het leven en werk van Cohen. 
% Informatie hierover is te vinden in boeken, artikels en op het internet.
% Gebruik minstens twee onafhankelijke bronnen en geef referenties naar de bronnen die je gebruikt. 
% Zorg ervoor dat uw beschrijving niet meer dan 30 regels bedraagt.
We geven hier een korte beschrijving van het leven en werk van Paul Cohen, een 20e-eeuwse wiskundige. 
Paul Cohen is geboren in 1934 in de Amerikaanse staat New Jersey.  
Hij doctoreerde aan de Universiteit van Chicago in 1958. 
In december 1963, op 29-jarige leeftijd, publiceerde hij \emph{The independence of the continuum hypothesis}, 
het eerste bewijs dat de continu\"um\-hypothese niet bewijsbaar is binnen de klassieke verzamelingenleer. 
De continu\"um\-hypothese heeft een mooie geschiedenis: in 1878 publiceerde Georg Cantor \emph{'Ein Beitrag zur Mannigfaltigkeitslehre'}, waarin hij de continu\"um\-hypothese formuleerde. In 1900 zette David Hilbert dit probleem (een nog onbewezen hypothese) op zijn beroemde lijst van 23 onopgeloste wiskundige problemen. Vervolgens bewees Kurt G\"odel in 1940 - Cohen werd 6 in dat jaar - dat er geen bewijs mogelijk is dat aantoont dat de continu\"um\-hypothese niet waar is. Met andere woorden: G\"odel toonde aan dat de negatie van de continu\"um\-hypothese niet kan bewezen worden. En vervolgens toonde Cohen in 1963 aan dat de continu\"um\-hypothese zelf ook niet kan bewezen worden. 
Cohen werd door G\"odel gecomplimenteerd voor zijn werk, en in 1966 won Cohen voor zijn werk over de continu\"um\-hypothese de Fieldsmedaille, een zeer belangrijke onderscheiding voor wiskundigen jonger dan 40 jaar. 
We komen nog terug op de continu\"um\-hypothese  in hoofdstuk \ref{sec_conthyp} van deze opdracht. 
Volgens Peter Sarnak heeft Cohen ook gewerkt aan de Riemann hypothese, een ander probleem dat voorkomt op de lijst van 23 problemen van Hilbert. De Riemann hypothese houdt verband met de verdeling van de priemgetallen, en is tot op heden niet bewezen. 
Paul Cohen overleed in Californi\"e in 2007 op 72-jarige leeftijd. 

Noot: de geraadpleegde bronnen voor dit stukje over Paul Cohen zijn:
\cite{wiki_en_cohen}, \cite{wiki_nl_cohen}, \cite{sarnak}, \cite{wiki_nl_fields}, \cite{wiki_en_cont}, \cite{wiki_en_hilb23}. 

%%%%%%%%%%%%%%%%%%%%%%%%%%%%%%%%%%%%%%%%%%%%%%%%%%%%%%%%%%%%%%%%%%%%%%%%%%%%%%%%%%%%%%%%%%%%%%%%%%%%%%%%%%%%%
\section{Continu\"umhypothese}
\label{sec_conthyp}
% Opdracht 2 Bewijs dat de continuumhypothese equivalent is met de volgende bewering:
% Als X een overaftelbare deelverzameling van R is, dan is er een injectieve functie van R naar X.

In dit hoofdstuk bewijzen we dat een alternatieve formulering van de continu\"um\-hypothese equivalent is met de formulering uit onze cursus (\cite{bew&red}). 

De continu\"um\-hypothese zoals geformuleerd in de cursus, beperkt tot deelverzamelingen van $\mathbb{R}$, luidt: 
\begin{equation}
\label{conthyp_1}
\neg ( \exists X \subset \mathbb{R} : \aleph_0 < |X| < 2^{\aleph_0} ) 
\end{equation}

Hierbij is $\aleph_0$ de kardinaliteit van $\mathbb{N}$, en is $2^{\aleph_0}$ de kardinaliteit van $\mathbb{R}$. 

Een alternatieve formulering is: 
\begin{equation}
\label{conthyp_alt}
X \subset \mathbb{R} \text{ is overaftelbaar } \implies \exists \text{ injectieve } f: \mathbb{R} \rightarrow X.
\end{equation}

We herschrijven eerst beide beweringen via logische regels. 

\eqref{conthyp_1} kunnen we ook schrijven als: 
\begin{equation}
\label{conthyp_2}
\forall X \subset \mathbb{R} : |X| \leq |\mathbb{N}| \lor |X| \geq |\mathbb{R}| 
\end{equation}

\eqref{conthyp_alt} kunnen we herschrijven via equivalentie van ($P \implies Q$) en ($\neg P \lor Q$): 
\begin{equation}
\label{conthyp_alt2}
\forall X \subset \mathbb{R} :  (X \text{ is niet overaftelbaar }) \lor (\exists \text{ injectieve } f: \mathbb{R} \rightarrow X)
\end{equation}

We bewijzen nu dat \eqref{conthyp_2} en \eqref{conthyp_alt2} equivalente beweringen zijn. 

\begin{proof}
Om de equivalentie te bewijzen, bewijzen we twee implicaties.  

Bewijs van \eqref{conthyp_2} $\implies$ \eqref{conthyp_alt2}: Neem $X$ een willekeurige deelverzameling van $\mathbb{R}$. Uit \eqref{conthyp_2} volgt dat er twee gevallen zijn. 
Geval 1: $|X| \leq |\mathbb{N}| $. 
Geval 2: $|X| \geq |\mathbb{R}|$. 
Voor geval 1, kunnen we rechtstreeks besluiten dat $X$ aftelbaar is, dus niet overaftelbaar, waarmee geval 1 bewezen is. 
Voor geval 2 volgt uit de definitie van $\leq$ voor kardinaliteiten dat er een injectieve functie bestaat van $\mathbb{R}$ naar $X$, waarmee geval 2 ook bewezen is. 

Bewijs van \eqref{conthyp_alt2} $\implies$ \eqref{conthyp_2}: : $X$ is een willekeurige deelverzameling van  $\mathbb{R}$, en er zijn ook hier twee gevallen. 
Geval 1: $X$ niet overaftelbaar. 
Geval 2: er bestaat een injectieve functie van $\mathbb{R}$ naar $X$. 
Voor geval 1 is gegeven dat $X$ niet overaftelbaar is, $X$ is bijgevolg aftelbaar, en is $|X|$ ofwel eindig ($|X| < |\mathbb{N}|$ of wel aftelbaar oneindig ($|X| = |\mathbb{N}|$), bijgevolg $|X| \leq |\mathbb{N}| $, waarmee geval 1 bewezen is. 
Voor geval 2 is gegeven dat er een injectieve functie bestaat van $\mathbb{R}$ naar $X$, waardoor we per definitie kunnen schrijven dat $|X| \geq |\mathbb{R}|$, waarmee ook dit geval bewezen is. 
\end{proof}

%%%%%%%%%%%%%%%%%%%%%%%%%%%%%%%%%%%%%%%%%%%%%%%%%%%%%%%%%%%%%%%%%%%%%%%%%%%%%%%%%%%%%%%%%%%%%%%%%%%%%%%%%%%%%
\section{Overaftelbare verzameling}
\label{sec_overaft}
% Opdracht 3 Formuleer de bewering dat P(N) equipotent is met 2^N als een stelling.
% Bewijs vervolgens deze stelling door een expliciete functie P(N) -> 2^N te definieen en te bewijzen dat dit een bijectie is.
In dit hoofdstuk bewijzen we volgende stelling: 

\begin{stelling}
\label{stelling_overaftelbaar}
$ P(\mathbb{N})$ is equipotent met $2^{\mathbb{N}} = \{ f \mid f : \mathbb{N} \rightarrow \{0,1\} \text{ is een functie}  \} $
\end{stelling}

De verzameling $2^{\mathbb{N}}$ kan gezien worden als de verzameling van alle oneindige rijen bestaande uit nullen en enen, waarbij de functiewaarde van elke natuurlijk getal te vinden is door te kijken naar de 0 of 1 op positie in de rij die overeenkomt met het natuurlijk getal. 
Bewijzen we nu stelling \ref{stelling_overaftelbaar}: 

\begin{proof}
We defini\"eren eerst een hulpfunctie $g_A$ voor elke element $A \in P(\mathbb{N})$ als volgt: 
\begin{equation}
  g_A: \mathbb{N} \rightarrow \{0,1\}: 
  g_A(x) = \left\{  
  \begin{aligned}
  0, &\text{ als } x \notin A \\
  1, &\text{ als } x \in A
  \end{aligned}
  \right.
\end{equation}

Het is duidelijk dat voor alle $A\in P(\mathbb{N})$ geldt: $g_A \in 2^{\mathbb{N}}$. 
Beschouwen we nu de volgende functie $f$ die gebruik maakt van voorgaande hulpfunctie: 
\begin{equation}
\label{fun_nat2_binstring}
f: P(\mathbb{N}) \rightarrow  2^{\mathbb{N}}: f(A) = g_A 
\end{equation}

Omdat we voor een willekeurige verzameling $A \subset \mathbb{N}$ een unieke rij nullen en enen $g_A$ kunnen construeren die aangeeft welke natuurlijke getallen in $A$ zitten, en welke niet, en dit maar op \'e\'en manier kan, is $f$ injectief. 
Omgekeerd kunnen we voor een willekeurige rij nullen en enen $g_A$ steeds een deelverzameling $A \subset \mathbb{N}$ vinden zodat $f(A)=g_A$, dus is $f$ surjectief. Bijgevolg is $f$ bijectief. 
Omdat we een bijectie hebben van $P(\mathbb{N})$ naar $2^{\mathbb{N}}$, zijn deze verzamelingen equipotent. 
\end{proof}

%%%%%%%%%%%%%%%%%%%%%%%%%%%%%%%%%%%%%%%%%%%%%%%%%%%%%%%%%%%%%%%%%%%%%%%%%%%%%%%%%%%%%%%%%%%%%%%%%%%%%%%%%%%%%
\section{Equipotentie \texorpdfstring{$P(\mathbb{N})$}{} en \texorpdfstring{$\mathbb{R}$}{}}
% bonusvraag: Bewijs dat P(N) equipotent is met R.

In dit hoofdstuk bewijzen we de volgende stelling: 
\begin{stelling}
\label{st_bonus}
$P(\mathbb{N})$ is equipotent met $\mathbb{R}$ 
\end{stelling}

We maken hierbij gemakshalve gebruik van de volgende lemmata, die we niet bewijzen, alleen kort toelichten.  

\begin{lemma}
\label{lemma_2N_int01} 
Er bestaat een bijectie $f: 2^{\mathbb{N}} \to \mathopen]0,1\mathclose[ $. 
\end{lemma}
Toelichting bij Lemma \ref{lemma_2N_int01}: inspiratie voor dit lemma haalden we uit \cite{youtube_penn}, waar een zeer gelijkaardige bewering wordt bewezen. 
Kort samengevat: we hadden eerder al aangegeven dat een element van $2^{\mathbb{N}}$ kan gezien worden als een oneindige rij nullen en enen. 
Schrijven we zo'n rij als  $(b_1, b_2, b_3, ...)$, dan kan men inzien dat uit dergelijke rij een re\"eel getal $x \in \mathopen]0,1\mathclose[$ berekend kan worden door de rij te beschouwen als een binaire ontwikkeling, en dat deze ontwikkeling uniek is voor elke $x$: 
\[ 
x = \sum_{n=1}^{\infty}{\frac{b_n}{2^n}} = 0,b_1 b_2 b_3 ...  \text{, met } b_n \in \{0,1\} 
\] 

\begin{lemma}
\label{lemma_int01_R} 
De functie $f: \mathopen]0,1\mathclose[ \to \mathbb{R} : f(x) = tan(\pi x - \frac{\pi}{2})$ is bijectief. 
\end{lemma}
Toelichting bij Lemma \ref{lemma_int01_R}: door naar de grafiek van deze functie te kijken, kan men inzien dat elke $x$ een unieke $f(x)$ geeft, en omgekeerd, en de functie bijgevolg bijectief is. 

Nu volgt het bewijs van Stelling \ref{st_bonus}. 
\begin{proof}
In hoofdstuk \ref{sec_overaft} hebben we bewezen dat er een bijectie $f: P(\mathbb{N}) \to  2^{\mathbb{N}}$ bestaat, zie vergelijking \eqref{fun_nat2_binstring}. 
Wegens Lemma \ref{lemma_2N_int01} bestaat er een bijectie 
$g: 2^{\mathbb{N}} \to \mathopen]0,1\mathclose[ $. 
Wegens Lemma \ref{lemma_int01_R} bestaat er een bijectie 
$h: \mathopen]0,1\mathclose[ \to \mathbb{R}$. 

Een samenstelling van drie bijecties is bijectief, bijgevolg is $(h \circ g \circ f): P(\mathbb{N}) \to  \mathbb{R}  $ een bijectie, en is $P(\mathbb{N})$ 
per definitie equipotent met $\mathbb{R}$ . 
\end{proof}




%%%%%%%%%%%%%%%%%%%%%%%%%%%%%%%%%%%%%%%%%%%%%%%%%%%%%%%%%%%%%%%%%%%%%%%%%%%%%%%%%%%%%%%%%%%%%%%%%%%%%%%%%%%%%
\begin{thebibliography}{9}
\bibitem{wiki_en_cohen} \url{https://en.wikipedia.org/wiki/Paul_Cohen}, Wikipedia Engels, geraadpleegd op 9/11/2023. 
\bibitem{wiki_nl_cohen} \url{https://nl.wikipedia.org/wiki/Paul_Cohen}, Wikipedia Nederlands, geraadpleegd op op 10/11/2023. 
\bibitem{sarnak} Peter Sarnak, Remembering Paul Cohen, Notices of the AMS, volume 57, nummer 7, Augustus 2010.
\bibitem{wiki_nl_fields}  \url{https://nl.wikipedia.org/wiki/Fieldsmedaille}, Wikipedia Nederlands, geraadpleegd op op 10/11/2023. 
\bibitem{wiki_en_cont} \url{https://en.wikipedia.org/wiki/Continuum_hypothesis}, Wikipedia Engels, geraadpleegd op op 10/11/2023. 
\bibitem{wiki_en_hilb23} \url{https://en.wikipedia.org/wiki/Hilbert%27s_problems}, Wikipedia Engels, geraadpleegd op op 10/11/2023.  
\bibitem{bew&red} Arno Kuijlaars, Bewijzen en Redeneren, cursustekst van KU Leuven, academiejaar 2023-2024. 
\bibitem{youtube_penn} Micahel Penn, Real Analysis $\mathbb{R}$ and $P(\mathbb{N})$, \url{https://www.youtube.com/watch?v=0fZSb23iFnk}.  


\end{thebibliography}


\end{document}



