\documentclass{article}
\usepackage{graphicx} % Required for inserting images
\usepackage{amsmath} 
\usepackage{amsfonts} % moet erbij om verzamelingen N,R... netjes te krijgen

% mark paragraphs with empty line instead of indented first line
\setlength{\parindent}{0em}
\setlength{\parskip}{1em}

\title{Nota's bewijzen en redeneren}
\author{Raf Meeusen}
\date{2023-2024}

\begin{document}

\maketitle

\section{Les 1, 27 sept}

\section{Oefenzitting 1, 4 okt}

Assistent Thomas. 

Oefening 1.1.2 gemaakt. Mijn oplossing: (v) voor eerste vraag (negatie), en (vii) voor tweede vraag (equivalent). Ik denk dat dit juist was. 

Onthouden: negatie van een "voor alle" is een "er bestaat". 

Oefening 1.4.2 gemaakt. Relatief makkelijk. 

Dan: nagedacht over onderscheid tussen ongerijmde en contrapositie. Niet altijd duidelijk. Principe van ongerijmde: stel dat het T.B. niet waar is, dan tonen we aan dat er een contradictie volgt. Bij contrapositie kan je gewoon vertrekken uit wat algemeenheden, dan bewijzen dat $\neg Q \implies \neg P$, en dan gewoon daaruit via de logica eigenschappen besluiten dat $P \implies Q$. 

Oefening 1.3.4 beginnen bekijken. Die 1.4 van ordening staat op p.8 van de cursus (strikte ongelijkheid wordt bewaard bij links en rechts met $c>0$ vermenigvuldigen. 

Oefening 1.5.1. Voor n=0 waar, dat heb ik nagegaan. Maar dan: inductiestap kon ik niet bewijzen, niet gevonden. 

Oefening 1.5.4a: niet helemaal uitgewerkt maar lijkt me perfect haalbaar. 

Oefening 1.5.4b: door Thomas op bord gezet. Hij gebruikt begrip inductiehypothese, en noemt dit IH en refereert hiernaar expliciet bij de inductiestap. 

Oefening 1.3.7: waarheidstabel eens opgesteld, zodat ik kolom had met 0 en 1 voor $P, Q, R$, $P \implies Q$, $Q \implies R$, $P \implies Q \land Q \implies R$, en tenslotte $P \implies R$. Dan nagekeken dat er onder  $P \implies R$ nergens een 0 staat waar $P \implies Q \land Q \implies R$ 1 is. 

Oefening 1.5.9. Thomas op bord iets gezet. c) zou aan te tonen zijn, maar b) niet. Iets met voldoende aannames. Zie foto. 
Zelf wat aan a) gewerkt: $a_2 = \frac{3}{2}$. $a_3 = \frac{7}{4}$... Maar niet verder aan gewerkt. 


\section{Les 2, 4 okt}

Hoofdstuk 2. Verzamelingen en kwantoren. Betekenis element ($x \in X$). Bekende verzamelingen $\mathbb{N}, \mathbb{Z}, \mathbb{Q}, \mathbb{R}, \mathbb{C}, \mathbb{N}_0, ... $
Deelverzameling $A \subset B$. Onze afspraak: deelverzameling mag ook gelijke verzameling zijn (niet bij iedereen de conventie, er bestaat ook een symbool $\subseteq$, maar laat ons dat niet gebruiken). De lege verzameling $\emptyset$. Opsomming elementen $\{a,b,c\}$. Via beperkende eigenschap: $\{x\in \mathbb{R} \mid x>2 \} $. Of een formule: $\{x^5 -3x  \mid  x \in \mathbb{Z} \} $

Bewerkingen met verzamelingen: $A \cup B$, $A \cap B$. Universum U, en complement-verzameling $A^C = U \setminus A$. Distributiviteit van doorsnede op unie. 

Eigenschap: complement van een doorsnede is gelijk aan unie van complementen: $(A \cap B)^C = A^C \cup B^C$. Kan je aantonen met Venn-diagram, en de vier mogelijkheden beschouwen (in A en in B, in A en niet in B, ...). Maar kan ook anders. Typische mogelijkheid voor gelijkheid van twee verzamelingen $A en B$, is in twee stappen. Stap 1: bewijs dat $A \subset B$, en stap 2: bewijs dat $B \subset A$. 

Voorbeeldbewijs in les gezien van $(A \cap B)^C = A^C \cup B^C$. Principe voor eerste inclusie: neem willekeurige $x$ uit $(A \cap B)^C$. Dan zit  $x$ niet in $A \cap B$. En dus geldt: $(x\notin A)$ of $(x\notin B)$. Dus geldt: $(x \in A^C)$ of $(x\in B^C)$. En daarom ... En dan nog tweede inclusie bewijzen. En klaar. 

De machtsverzameling van verzameling $X$, genoteerd als $P(X)$, verzameling van alle deelverzamelingen. Noot: wij noteerden dat altijd $2^X$. Tricky: $P(\emptyset) = \{ \emptyset \}$, dus niet gewoon gelijk aan de lege verzameling. 

Cartesisch product van twee verzamelingen: $X \times Y$. 

Algemene unies en doorsnedes: unie of doorsnede gebaseerd op verzamelingen met een index. 

Kwantoren: "voor elke": $\forall$  en "er bestaat" $\exists$. Men noemt ze respectievelijk  de universele en de existenti\"ele kwantor. Dan in les even op andere manier geschreven (verzameling gelijk aan A voor de universele, en verzameling gelijk aan lege verzameling voor de existenti\"ele). Dat blijkt wel handig om bvb. volgende uitspraak als waar of onwaar te bekijken: $\forall x \in \emptyset : x>2$. Die is niet evident, en die is waar, want als je het herschrijft in die andere vorm, klopt het helemaal. 
Daarna ook nog andere uitdrukking met implicatie bekeken. Mmm, opnieuw implicaties, blijft lastig om over na te denken. Als linkerkant niet waar is, is de implicatie altijd waar. (maar daarom niet de rechterkant!)... 

Combinaties kwantoren. Volgorde is van belang. 

Negatie ($\lnot$) van de $\forall$ wordt een $\exists$. Negatie van de $\exists$ wordt een $\forall$. 



\section{KEEP ME TO JUMP TO END}

\end{document}