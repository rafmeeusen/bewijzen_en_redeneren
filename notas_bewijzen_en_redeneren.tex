\documentclass{article}
\usepackage{graphicx} % Required for inserting images
\usepackage{amsmath} 

% mark paragraphs with empty line instead of indented first line
\setlength{\parindent}{0em}
\setlength{\parskip}{1em}

\title{Nota's bewijzen en redeneren}
\author{Raf Meeusen}
\date{2023-2024}

\begin{document}

\maketitle

\section{Les 1, 27 sept}

\section{Oefenzitting 1, 4 okt}

Assistent Thomas. 

Oefening 1.1.2 gemaakt. Mijn oplossing: (v) voor eerste vraag (negatie), en (vii) voor tweede vraag (equivalent). Ik denk dat dit juist was. 

Onthouden: negatie van een "voor alle" is een "er bestaat". 

Oefening 1.4.2 gemaakt. Relatief makkelijk. 

Dan: nagedacht over onderscheid tussen ongerijmde en contrapositie. Niet altijd duidelijk. Principe van ongerijmde: stel dat het T.B. niet waar is, dan tonen we aan dat er een contradictie volgt. Bij contrapositie kan je gewoon vertrekken uit wat algemeenheden, dan bewijzen dat $\neg Q \implies \neg P$, en dan gewoon daaruit via de logica eigenschappen besluiten dat $P \implies Q$. 

Oefening 1.3.4 beginnen bekijken. Die 1.4 van ordening staat op p.8 van de cursus (strikte ongelijkheid wordt bewaard bij links en rechts met $c>0$ vermenigvuldigen. 

Oefening 1.5.1. Voor n=0 waar, dat heb ik nagegaan. Maar dan: inductiestap kon ik niet bewijzen, niet gevonden. 

Oefening 1.5.4a: niet helemaal uitgewerkt maar lijkt me perfect haalbaar. 

Oefening 1.5.4b: door Thomas op bord gezet. Hij gebruikt begrip inductiehypothese, en noemt dit IH en refereert hiernaar expliciet bij de inductiestap. 

Oefening 1.3.7: waarheidstabel eens opgesteld, zodat ik kolom had met 0 en 1 voor $P, Q, R$, $P \implies Q$, $Q \implies R$, $P \implies Q \land Q \implies R$, en tenslotte $P \implies R$. Dan nagekeken dat er onder  $P \implies R$ nergens een 0 staat waar $P \implies Q \land Q \implies R$ 1 is. 

Oefening 1.5.9. Thomas op bord iets gezet. c) zou aan te tonen zijn, maar b) niet. Iets met voldoende aannames. Zie foto. 
Zelf wat aan a) gewerkt: $a_2 = \frac{3}{2}$. $a_3 = \frac{7}{4}$... Maar niet verder aan gewerkt. 


\section{Les 2, 4 okt}




\section{KEEP ME TO JUMP TO END}

\end{document}