\documentclass{article}
\usepackage{graphicx} % Required for inserting images
\usepackage{amsmath} 
\usepackage{amsfonts} % moet erbij om verzamelingen N,R... netjes te krijgen

% mark paragraphs with empty line instead of indented first line
\setlength{\parindent}{0em}
\setlength{\parskip}{1em}

\title{Oefeningen bewijzen en redeneren}
\author{Raf Meeusen}
\date{2023-2024}

\begin{document}

\maketitle

\section*{Oefenzitting 1, 4 okt}

Assistent Thomas. 

Oefening 1.1.2 gemaakt. Mijn oplossing: (v) voor eerste vraag (negatie), en (vii) voor tweede vraag (equivalent). Ik denk dat dit juist was. 

Onthouden: negatie van een "voor alle" is een "er bestaat". 

Oefening 1.4.2 gemaakt. Relatief makkelijk. 

Dan: nagedacht over onderscheid tussen ongerijmde en contrapositie. Niet altijd duidelijk. Principe van ongerijmde: stel dat het T.B. niet waar is, dan tonen we aan dat er een contradictie volgt. Bij contrapositie kan je gewoon vertrekken uit wat algemeenheden, dan bewijzen dat $\neg Q \implies \neg P$, en dan gewoon daaruit via de logica eigenschappen besluiten dat $P \implies Q$. 

Oefening 1.3.4 beginnen bekijken. Die 1.4 van ordening staat op p.8 van de cursus (strikte ongelijkheid wordt bewaard bij links en rechts met $c>0$ vermenigvuldigen. 

Oefening 1.5.1. Voor n=0 waar, dat heb ik nagegaan. Maar dan: inductiestap kon ik niet bewijzen, niet gevonden. 

Oefening 1.5.4a: niet helemaal uitgewerkt maar lijkt me perfect haalbaar. 

Oefening 1.5.4b: door Thomas op bord gezet. Hij gebruikt begrip inductiehypothese, en noemt dit IH en refereert hiernaar expliciet bij de inductiestap. 

Oefening 1.3.7: waarheidstabel eens opgesteld, zodat ik kolom had met 0 en 1 voor $P, Q, R$, $P \implies Q$, $Q \implies R$, $P \implies Q \land Q \implies R$, en tenslotte $P \implies R$. Dan nagekeken dat er onder  $P \implies R$ nergens een 0 staat waar $P \implies Q \land Q \implies R$ 1 is. 

Oefening 1.5.9. Thomas op bord iets gezet. c) zou aan te tonen zijn, maar b) niet. Iets met voldoende aannames. Zie foto. 
Zelf wat aan a) gewerkt: $a_2 = \frac{3}{2}$. $a_3 = \frac{7}{4}$... Maar niet verder aan gewerkt. 


\section*{Oefenzitting 2, 11 okt}
\subsection*{Voorbereidingsoefeningen voor thuis} 

Oefening 2.1.2. Bewijs onderdeel (c) van Stelling 2.1.10 met behulp van Venndiagrammen.
(niet afgemaakt, maar zou wel lukken) 
Deel a) $A \cup (B \cap C) = (A \cup B) \cap (A \cup C) $
Venn-diagram: met 8 gebieden (1 buiten alle verzamelingen, 1 doorsnede van de drie, 3 zonder doorsnede, 3 doorsnedes twee-aan-twee). 
Dan nagaan dat voor elk gebied geldt dat wat links zit, ook rechts zit, en wat rechts zit, ook links zit. 

Deel b) $A \cap (B \cup C) = (A \cap B) \cup (A \cap C)$
Gelijkaardig. 

Oefening 2.1.3. Bewijs onderdeel (d) van Stelling 2.1.10 met behulp van waarheidstabellen.
(de wetten van De Morgan)
Deel a) $(A \cup B)^c = A^c \cap B^c$ 

Deel b) $(A \cap B)^c = A^c \cup B^c$ 

Bewijs met waarheidstabellen: niet gemaakt, maar zou wel lukken denk ik. 

Oefening 2.2.1. Onderzoek of de volgende beweringen waar of niet waar zijn en geef een
bewijs.
\begin{itemize}
\item a) niet waar, bvb. $m=2$ en $n=1$
\item b) waar ; bewijs is ook voorbeeld nemen
\item c) waar ; bewijs: kies n=m 
\item d) niet waar; bewijs: ik vind altijd een n waarvoor het niet klopt
\item e) waar
\item f) niet waar

\end{itemize}

Oefening 2.2.7. Een deelverzameling $A \subset R$ is naar boven begrensd als ... Geef een negatie van deze bewering. 

OK, eens gemaakt. Moet je echt stap voor stap doen, negatie eerst binnen brengen in eerste kwantor, dan in tweede kwantor. Pas in tweede stap wordt propositie omgekeerd naar $a>x$. 

\subsection{Oefenzitting}

Oefening 2.1.1. Bewijs de eerste uitspraak van onderdeel (c) van Stelling 2.1.10 met een logische redenering.
Deel a) $A \cup (B \cap C) = (A \cup B) \cap (A \cup C) $. Oefening gemaakt voor ene inclusie: voor een $x$ in linkse verzameling geldt dat die ook in de rechtse zit (x zit in A of in B doorsnede C, gevalsonderscheiding, geval x in A, dan ook in A unie B, en ook in A unie C, dus ook in de doorsnede). Etc. etc. 
Ene richting bewijs is de truuk om uit te breiden met unie, wat altijd mag (als x in A zit, dan ook in A unie whatever). Dan ook omgekeerd. Is wat moeilijker, vind ik. 

2.1.4 (b) bewijs equivalentie 
$ A \subset B \Leftrightarrow A \cap B = A$
Bewijs in twee grote stappen, de twee implicaties. Implicatie van links naar rechts: stel dat A deelverzameling van B, dan zou volgen dat $A \cap B = A$. Deze gelijkheid van verzamelingen tonen we aan met twee inclusies: $A \cap B \subset A$ en $A \subset A \cap B $. 
Eerste inclusie is altijd geldig, tweede inclusie geldt omdat we weten dat $ A \subset B$. Te redeneren met willekeurige x in $A$ etc. etc. 

Oefening 2.1.5 i.v.m. symmetrisch verschil. a) bewijs dat gelijk is aan unie van verschilverzamelingen. Zal wel gelijkaardig zijn als andere bewijzen met verzamelingen. Niet helemaal uitgewerkt maar wel over nagedacht. b) wanneer is deze leeg? Antwoord: als $A=B$. Om te bewijzen: als $A=B$ dan $A \Delta B = \emptyset$. Niet zo evident. Wanneer is een $A \setminus B $ leeg? Alleen als $A \subset B$. 
c) Dit is het geval als B leeg is. 
Oplossing op bord in oefenzitting voor a) maakt gebruik van eigenschap $X \setminus Y = X \cap Y^c$. Dus lijkt me een tip: veel eigenschappen van verzamelingen proberen te gebruiken bij zo'n bewijs. Deel a) kan dus bewezen worden door helemaal uit te schrijven en te herschrijven vanuit definitie. 


Oefening 2.1.9a: gemaakt. Eerst alle drie verzamelingen koppels uitgeschreven: $A \times (B \cup C) = \{ (a,x) | ...$. Dan door twee inclusies aangetoond dat links en rechts hetzelfde zijn. 
Oefening 2.1.9b: niet meer gemaakt. 
Oefening 2.1.9c: niet meer gemaakt. 

Oefening 2.1.10: OK, dat lukte me wel. Gewoon aangetoond dat willekeurig koppel uit linkse verzameling altijd in de rechtse zit. 

Oefening 2.2.2 per ongeluk begonnen:
\begin{itemize}
    \item b) niet waar, er is voor elke x maar 1 y die eraan voldoet
    \item c) waar
\end{itemize}

Oefening 2.2.6: Antwoord is Ja. op bord gestaan in oefenzitting. Eigenlijk gewoon in tekst geschreven hoe je dit redeneert/inziet. Nogal triviaal, via aan $a_0$ die bestaat, en die je dan gebruikt om het aan te tonen. 


Oefening 2.2.9 p. 36. Gemaakt na oefenzitting. Kan ik wel denk ik. Onthouden dat negatie van een implicatie als volgt is: $\neg (P \implies Q) \Leftrightarrow P \land \neg Q$, wat ook eenvoudig te zien is in waarheidstabel. 

Oefening 2.2.10 p. 37: Gemaakt na oefenzitting. Beetje onduidelijk of hele bewering "als dan" moet genomen worden, of alleen stuk met $\exists$. Nog eens herhaald wat een predicaat/propositie juist is (iets dat waar of niet waar is, al dan niet met variabelen). 

Oefening 2.2.11: beginnen maken, maar echt moeilijk. Dan gezien dat ze niet in de lijst stond, maar dat het 2.2.12 was dat ik moest maken. 

Oefening 2.2.12: nog niet gemaakt. 


\section*{Oefenzitting 3, 18 okt}

\subsection*{Voorbereiding}

Oefening 3.1.2 b p.45: $O^{-1}$ is de relatie "kind van". $O^{-1} \circ O$ is dan "heeft dezelfde kinderen als", of ook: "is zichzelf of mede-ouder".  

Oefening 3.1.2 d: $O \circ Z = \emptyset$

Oefening 3.1.3 (a) p.45: $R^{-1} \circ R$ is "heeft een vak gemeenschappelijk met" , "zit samen in de les met student...". 

Oefening 3.1.8 p. 45: snap niet goed wat bedoeling is; lijkt triviaal en al in theorie op bord gedaan. 

Oefening 3.2.1 p. 49: 

\begin{tabular}{l|c|c|c}
   & Refl & Symm & Trans\\
a) & x   &   x   &   \\
b) & x   &       & x \\
c) &     &   x   &   \\
d) & x   &   x   & x \\
e) &     &   x   & x \\
\end{tabular}

Oefening 3.2.3 (a) p. 49: 
Definities eerst eens opgeschreven. Heel gelijkaardig maar niet hetzelfde. 
Redenering opgeschreven: als $R$ reflexief is, dan geldt $\forall x \in X$ dat $(x,x) \in R$. Dus alle elementen van $I_x$ zitten in $R$ want ze voldoen allemaal aan de vorm $(x,x)$. 
Omgekeerde pijl: als $I_x \subset R$, dan zitten alle elementen van $I_x$ in de relatie $R$, dus voor alle $x$ geldt dat $(x,x)$ in $R$ zit, dus dan is R reflexief. 

Oefening 3.2.9 p. 50: 
\begin{itemize}
    \item a) beginnen met zelfde letter: is equival.rel. ; klassen [a], [b], ...
    \item b) hebben min. 1 letter gemeenschappelijk; NIET; is niet transitief; bvb. 'kat' $\sim$ 'ket' $\sim$ 'en'. 
    \item c) hebben evenveel letters: is equival. rel.; klassen [1], [2], [3], ... 
\end{itemize}

\subsection*{oefenzitting van 18 oktober} 

Oefening 3.1.2 (e), p. 45: "heeft als dochter" ; opgelet, $Z^{-1}$ is \emph{niet} "is zus van", en \emph{ook niet} "is broer of zus van", want de relatie gaat alleen naar meisjes; dus $Z^{-1}$ zou kunnen zijn "is zuszegger van". 

Oefening 3.1.3 (b) p. 45: (zie p.43 voor R en S); $R^{-1} \circ \S^{-1}$: "prof p geeft les aan student q". 

Oefening 3.1.7, p. 45: op papier wat geschetst; ik zou beginnen met definities van inverse voor $R^{-1}$ en  $S^{-1}$, waaruit dan volgt (via definitie samenstelling): $R^{-1} \circ S^{-1} = \{(z,x) \in Z \times X | (z,y) \in S^{-1} en (y,x) \in R^{-1}   \}$. Dan inverse van deze relatie uitschrijven, wat koppels $(x,z)$ geeft, en dat na herschrijven uitkomt op $S \circ R$. Dus als inverse van rechterlid van T.B. gelijk is aan inverse van linkerlid, dan is T.B. bewezen. 

Oefening 3.2.3 (c), p. 49: over nagedacht en wat neergeschreven, maar geen sluitend bewijs proberen te formuleren. Principe was: de twee implicaties afzonderlijk te bewijzen; als transitief, dan is samenstelling deelverzameling; en omgekeerd, als samenstelling deelverzameling is, dan is relatie transitief. 

Oefening 3.2.6 (p. 50):  
\begin{itemize}
    \item a) waar; voor elke $x \in X$ geldt $(x,x) \in R $ en $(x,x) \in S $. Dus elke R-pijl die eindigt in een bepaalde $x$, wordt gevolgd door een S-pijl die begint in $x$ en ook weer eindigt in $x$, dus is samenstelling ook reflexief
    \item b) niet waar; stel $(x,y), (y,x) \in R$ en $(y,z),(z,y) \in S$. Dan zit $(x,z)$ in $S \circ R$. Maar $(z,x)$ niet.
    \item c) niet waar; mmmm best moeilijke; wat met drie kleuren getekend, en tegenvoorbeeld gevonden; 
\end{itemize}

Oefening 3.2.10 (p.50): 
\begin{itemize}
    \item a) niet ; niet symmetrisch
    \item b) ja; transitiviteit: je kan voor $(x,z)$ zien dat $x-z$ ook rationaal is als ...
    \item c) ja; (kijk naar $10^{-n}$ voor symmetrisch, en naar $10^{m+n}$ voor transitief
\end{itemize}

\subsection*{Extra oefeningen} 

3.1.6, 
3.2.12 (als u het begrip vectorruimte kent)
3.3.2, 
3.3.3 


\section*{Oefenzitting 4, 25 okt} 


\subsection*{Voorbereiding}

4.1.3, 4.1.9, 4.1.11 (a,b,c), 4.2.2, 4.2.6, 4.2.9 



\subsection*{oefenzitting} 

4.1.4, 4.1.7 (c,d), 4.1.11 (d,e,f), 4.1.12 (a,b,c), 4.2.4, 4.2.7, 4.2.10, 4.2.13 

\subsection{Extra oefeningen}

 4.1.8, 4.1.10, 4.2.8, 4.2.12, 4.2.14,

3.1.2 (b,d), 3.1.3 (a), 3.1.8, 3.2.1, 3.2.3 (a), 3.2.9




\section*{NEXT}

\end{document}