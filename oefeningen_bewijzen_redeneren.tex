\documentclass{article}
\usepackage{graphicx} % Required for inserting images
\usepackage{amsmath, amsthm} 
\usepackage{amsfonts} % moet erbij om verzamelingen N,R... netjes te krijgen

\renewcommand*{\proofname}{Bewijs}


% mark paragraphs with empty line instead of indented first line
\setlength{\parindent}{0em}
\setlength{\parskip}{1em}

\title{Oefeningen bewijzen en redeneren}
\author{Raf Meeusen}
\date{2023-2024}

\begin{document}

\maketitle

\section*{Oefenzitting 1, 4 okt}

Assistent Thomas. 

Oefening 1.1.2 gemaakt. Mijn oplossing: (v) voor eerste vraag (negatie), en (vii) voor tweede vraag (equivalent). Ik denk dat dit juist was. 

Onthouden: negatie van een "voor alle" is een "er bestaat". 

Oefening 1.4.2 gemaakt. Relatief makkelijk. 

Dan: nagedacht over onderscheid tussen ongerijmde en contrapositie. Niet altijd duidelijk. Principe van ongerijmde: stel dat het T.B. niet waar is, dan tonen we aan dat er een contradictie volgt. Bij contrapositie kan je gewoon vertrekken uit wat algemeenheden, dan bewijzen dat $\neg Q \implies \neg P$, en dan gewoon daaruit via de logica eigenschappen besluiten dat $P \implies Q$. 

Oefening 1.3.4 beginnen bekijken. Die 1.4 van ordening staat op p.8 van de cursus (strikte ongelijkheid wordt bewaard bij links en rechts met $c>0$ vermenigvuldigen. 

Oefening 1.5.1. Voor n=0 waar, dat heb ik nagegaan. Maar dan: inductiestap kon ik niet bewijzen, niet gevonden. 

Oefening 1.5.4a: niet helemaal uitgewerkt maar lijkt me perfect haalbaar. 

Oefening 1.5.4b: door Thomas op bord gezet. Hij gebruikt begrip inductiehypothese, en noemt dit IH en refereert hiernaar expliciet bij de inductiestap. 

Oefening 1.3.7: waarheidstabel eens opgesteld, zodat ik kolom had met 0 en 1 voor $P, Q, R$, $P \implies Q$, $Q \implies R$, $P \implies Q \land Q \implies R$, en tenslotte $P \implies R$. Dan nagekeken dat er onder  $P \implies R$ nergens een 0 staat waar $P \implies Q \land Q \implies R$ 1 is. 

Oefening 1.5.9. Thomas op bord iets gezet. c) zou aan te tonen zijn, maar b) niet. Iets met voldoende aannames. Zie foto. 
Zelf wat aan a) gewerkt: $a_2 = \frac{3}{2}$. $a_3 = \frac{7}{4}$... Maar niet verder aan gewerkt. 


\section*{Oefenzitting 2, 11 okt}
\subsection*{Voorbereidingsoefeningen voor thuis} 

Oefening 2.1.2. Bewijs onderdeel (c) van Stelling 2.1.10 met behulp van Venndiagrammen.
(niet afgemaakt, maar zou wel lukken) 
Deel a) $A \cup (B \cap C) = (A \cup B) \cap (A \cup C) $
Venn-diagram: met 8 gebieden (1 buiten alle verzamelingen, 1 doorsnede van de drie, 3 zonder doorsnede, 3 doorsnedes twee-aan-twee). 
Dan nagaan dat voor elk gebied geldt dat wat links zit, ook rechts zit, en wat rechts zit, ook links zit. 

Deel b) $A \cap (B \cup C) = (A \cap B) \cup (A \cap C)$
Gelijkaardig. 

Oefening 2.1.3. Bewijs onderdeel (d) van Stelling 2.1.10 met behulp van waarheidstabellen.
(de wetten van De Morgan)
Deel a) $(A \cup B)^c = A^c \cap B^c$ 

Deel b) $(A \cap B)^c = A^c \cup B^c$ 

Bewijs met waarheidstabellen: niet gemaakt, maar zou wel lukken denk ik. 

Oefening 2.2.1. Onderzoek of de volgende beweringen waar of niet waar zijn en geef een
bewijs.
\begin{itemize}
\item a) niet waar, bvb. $m=2$ en $n=1$
\item b) waar ; bewijs is ook voorbeeld nemen
\item c) waar ; bewijs: kies n=m 
\item d) niet waar; bewijs: ik vind altijd een n waarvoor het niet klopt
\item e) waar
\item f) niet waar

\end{itemize}

Oefening 2.2.7. Een deelverzameling $A \subset R$ is naar boven begrensd als ... Geef een negatie van deze bewering. 

OK, eens gemaakt. Moet je echt stap voor stap doen, negatie eerst binnen brengen in eerste kwantor, dan in tweede kwantor. Pas in tweede stap wordt propositie omgekeerd naar $a>x$. 

\subsection{Oefenzitting}

Oefening 2.1.1. Bewijs de eerste uitspraak van onderdeel (c) van Stelling 2.1.10 met een logische redenering.
Deel a) $A \cup (B \cap C) = (A \cup B) \cap (A \cup C) $. Oefening gemaakt voor ene inclusie: voor een $x$ in linkse verzameling geldt dat die ook in de rechtse zit (x zit in A of in B doorsnede C, gevalsonderscheiding, geval x in A, dan ook in A unie B, en ook in A unie C, dus ook in de doorsnede). Etc. etc. 
Ene richting bewijs is de truuk om uit te breiden met unie, wat altijd mag (als x in A zit, dan ook in A unie whatever). Dan ook omgekeerd. Is wat moeilijker, vind ik. 

2.1.4 (b) bewijs equivalentie 
$ A \subset B \Leftrightarrow A \cap B = A$
Bewijs in twee grote stappen, de twee implicaties. Implicatie van links naar rechts: stel dat A deelverzameling van B, dan zou volgen dat $A \cap B = A$. Deze gelijkheid van verzamelingen tonen we aan met twee inclusies: $A \cap B \subset A$ en $A \subset A \cap B $. 
Eerste inclusie is altijd geldig, tweede inclusie geldt omdat we weten dat $ A \subset B$. Te redeneren met willekeurige x in $A$ etc. etc. 

Oefening 2.1.5 i.v.m. symmetrisch verschil. a) bewijs dat gelijk is aan unie van verschilverzamelingen. Zal wel gelijkaardig zijn als andere bewijzen met verzamelingen. Niet helemaal uitgewerkt maar wel over nagedacht. b) wanneer is deze leeg? Antwoord: als $A=B$. Om te bewijzen: als $A=B$ dan $A \Delta B = \emptyset$. Niet zo evident. Wanneer is een $A \setminus B $ leeg? Alleen als $A \subset B$. 
c) Dit is het geval als B leeg is. 
Oplossing op bord in oefenzitting voor a) maakt gebruik van eigenschap $X \setminus Y = X \cap Y^c$. Dus lijkt me een tip: veel eigenschappen van verzamelingen proberen te gebruiken bij zo'n bewijs. Deel a) kan dus bewezen worden door helemaal uit te schrijven en te herschrijven vanuit definitie. 


Oefening 2.1.9a: gemaakt. Eerst alle drie verzamelingen koppels uitgeschreven: $A \times (B \cup C) = \{ (a,x) | ...$. Dan door twee inclusies aangetoond dat links en rechts hetzelfde zijn. 
Oefening 2.1.9b: niet meer gemaakt. 
Oefening 2.1.9c: niet meer gemaakt. 

Oefening 2.1.10: OK, dat lukte me wel. Gewoon aangetoond dat willekeurig koppel uit linkse verzameling altijd in de rechtse zit. 

Oefening 2.2.2 per ongeluk begonnen:
\begin{itemize}
    \item b) niet waar, er is voor elke x maar 1 y die eraan voldoet
    \item c) waar
\end{itemize}

Oefening 2.2.6: Antwoord is Ja. op bord gestaan in oefenzitting. Eigenlijk gewoon in tekst geschreven hoe je dit redeneert/inziet. Nogal triviaal, via aan $a_0$ die bestaat, en die je dan gebruikt om het aan te tonen. 


Oefening 2.2.9 p. 36. Gemaakt na oefenzitting. Kan ik wel denk ik. Onthouden dat negatie van een implicatie als volgt is: $\neg (P \implies Q) \Leftrightarrow P \land \neg Q$, wat ook eenvoudig te zien is in waarheidstabel. 

Oefening 2.2.10 p. 37: Gemaakt na oefenzitting. Beetje onduidelijk of hele bewering "als dan" moet genomen worden, of alleen stuk met $\exists$. Nog eens herhaald wat een predicaat/propositie juist is (iets dat waar of niet waar is, al dan niet met variabelen). 

Oefening 2.2.11: beginnen maken, maar echt moeilijk. Dan gezien dat ze niet in de lijst stond, maar dat het 2.2.12 was dat ik moest maken. 

Oefening 2.2.12: nog niet gemaakt. 


\section*{Oefenzitting 3, 18 okt}

\subsection*{Voorbereiding}

Oefening 3.1.2 b p.45: $O^{-1}$ is de relatie "kind van". $O^{-1} \circ O$ is dan "heeft dezelfde kinderen als", of ook: "is zichzelf of mede-ouder".  

Oefening 3.1.2 d: $O \circ Z = \emptyset$

Oefening 3.1.3 (a) p.45: $R^{-1} \circ R$ is "heeft een vak gemeenschappelijk met" , "zit samen in de les met student...". 

Oefening 3.1.8 p. 45: snap niet goed wat bedoeling is; lijkt triviaal en al in theorie op bord gedaan. 

Oefening 3.2.1 p. 49: 

\begin{tabular}{l|c|c|c}
   & Refl & Symm & Trans\\
a) & x   &   x   &   \\
b) & x   &       & x \\
c) &     &   x   &   \\
d) & x   &   x   & x \\
e) &     &   x   & x \\
\end{tabular}

Oefening 3.2.3 (a) p. 49: 
Definities eerst eens opgeschreven. Heel gelijkaardig maar niet hetzelfde. 
Redenering opgeschreven: als $R$ reflexief is, dan geldt $\forall x \in X$ dat $(x,x) \in R$. Dus alle elementen van $I_x$ zitten in $R$ want ze voldoen allemaal aan de vorm $(x,x)$. 
Omgekeerde pijl: als $I_x \subset R$, dan zitten alle elementen van $I_x$ in de relatie $R$, dus voor alle $x$ geldt dat $(x,x)$ in $R$ zit, dus dan is R reflexief. 

Oefening 3.2.9 p. 50: 
\begin{itemize}
    \item a) beginnen met zelfde letter: is equival.rel. ; klassen [a], [b], ...
    \item b) hebben min. 1 letter gemeenschappelijk; NIET; is niet transitief; bvb. 'kat' $\sim$ 'ket' $\sim$ 'en'. 
    \item c) hebben evenveel letters: is equival. rel.; klassen [1], [2], [3], ... 
\end{itemize}

\subsection*{oefenzitting van 18 oktober} 

Oefening 3.1.2 (e), p. 45: "heeft als dochter" ; opgelet, $Z^{-1}$ is \emph{niet} "is zus van", en \emph{ook niet} "is broer of zus van", want de relatie gaat alleen naar meisjes; dus $Z^{-1}$ zou kunnen zijn "is zuszegger van". 

Oefening 3.1.3 (b) p. 45: (zie p.43 voor R en S); $R^{-1} \circ \S^{-1}$: "prof p geeft les aan student q". 

Oefening 3.1.7, p. 45: op papier wat geschetst; ik zou beginnen met definities van inverse voor $R^{-1}$ en  $S^{-1}$, waaruit dan volgt (via definitie samenstelling): $R^{-1} \circ S^{-1} = \{(z,x) \in Z \times X | (z,y) \in S^{-1} en (y,x) \in R^{-1}   \}$. Dan inverse van deze relatie uitschrijven, wat koppels $(x,z)$ geeft, en dat na herschrijven uitkomt op $S \circ R$. Dus als inverse van rechterlid van T.B. gelijk is aan inverse van linkerlid, dan is T.B. bewezen. 

Oefening 3.2.3 (c), p. 49: over nagedacht en wat neergeschreven, maar geen sluitend bewijs proberen te formuleren. Principe was: de twee implicaties afzonderlijk te bewijzen; als transitief, dan is samenstelling deelverzameling; en omgekeerd, als samenstelling deelverzameling is, dan is relatie transitief. 

Oefening 3.2.6 (p. 50):  
\begin{itemize}
    \item a) waar; voor elke $x \in X$ geldt $(x,x) \in R $ en $(x,x) \in S $. Dus elke R-pijl die eindigt in een bepaalde $x$, wordt gevolgd door een S-pijl die begint in $x$ en ook weer eindigt in $x$, dus is samenstelling ook reflexief
    \item b) niet waar; stel $(x,y), (y,x) \in R$ en $(y,z),(z,y) \in S$. Dan zit $(x,z)$ in $S \circ R$. Maar $(z,x)$ niet.
    \item c) niet waar; mmmm best moeilijke; wat met drie kleuren getekend, en tegenvoorbeeld gevonden; 
\end{itemize}

Oefening 3.2.10 (p.50): 
\begin{itemize}
    \item a) niet ; niet symmetrisch
    \item b) ja; transitiviteit: je kan voor $(x,z)$ zien dat $x-z$ ook rationaal is als ...
    \item c) ja; (kijk naar $10^{-n}$ voor symmetrisch, en naar $10^{m+n}$ voor transitief
\end{itemize}

\subsection*{Extra oefeningen} 

3.1.6, 
3.2.12 (als u het begrip vectorruimte kent)
3.3.2, 
3.3.3 


\section*{Oefenzitting 4, 25 okt} 


\subsection*{Voorbereiding}

Oef. 4.1.3 (p.57): $f(\pi)=3$, en $f(-\pi)=-4$. Functieomschrijving: rond naar beneden af op geheel getal. In principe om aan te tonen dat $f$ een functie is: enerzijds existentie aantonen van een $n$ voor elke $x$, anderzijds uniciteit van dit beeld aantonen voor elke $x$. Heb ik niet gedaan, dus oefening niet helemaal gemaakt eigenlijk. 


Oef. 4.1.9 p. 58: bij a) zijn er 3 koppels in de elke functie, elk koppel heeft willekeurige $y \in \{ a,b \}$, dus twee mogelijke $y$-waarden. Dus: 2x2x2 = 8 functies. Bij b) zijn er 2 koppels, telkens 3 $y$-waarden mogelijk, dus $3^2=9$ functies. 

Oef. 4.1.11 (a,b,c) op p. 58: 
a) is juist; heb ik beredeneerd door naar definities te kijken, en ik kwam eigenlijk uit dat de verzamelingen gelijk zijn, dus dat b) en c) ook waar zijn. MAAR: heel goed mogelijk dat ik fout heb gemaakt! 
Moet dus eigenlijk formeler. En ik heb dus niet meer gekeken naar b) en c). In oefenzitting gevraagd: het klopt, alle drie a, b, c zijn waar hier. 


Oef. 4.2.2 (p. 63): 
Even recap: injectie is "geen dubbels, geen twee duiven in 1 kotje". En surjectie is "geen lege y-waarden". 

\begin{tabular}{c|c|c|c}
      &  inj. & surj. & bij. \\
$f_1$ &       &       &    \\
$f_2$ &   X   &       &    \\
$f_3$ &       &   X   &    \\
$f_4$ &   X   &   X   & X 
\end{tabular}

Oef. 4.2.6 (p.64): a) 6, b) 0, c) 0. 


Oef. 4.2.9 (p.64): begonnen, maar ik vond geen voorschrift voor $f^{-1}$, dus aan Thomas gaan vragen. Dan ingezien dat ik wat harder had moeten proberen. 
Als $y = \frac{3x}{x-2}$, dan is $yx-2y=3x$, en $x= \frac{-2y}{3-y}$, dus dit is voorschrift van $f^{-1}$. Thomas deed het ietsje anders qua notatie: vertrekpunt $f(f^{-1}(y)) = y$, dus als je $f^{-1}$ in vult in $f(x)$ en gelijkstelt aan $y$, en dan oplost naar $f^{-1}$, dan kom je het ook uiteraard. 
$Y = \mathbb{R} \setminus \{ 3 \}$


\subsection*{oefenzitting} 

Oef. 4.1.4 (p.57):  
voorbeeld dat op bord heeft gestaan. Neem $X=\{x_1\}, Y=\{y_1, y_2\}, Z= \{z_1  \} $, en $R=\{(x_1, y_1), (x_1, y_2) \}$, en $S =\{(y_2, z_1) \}  $. Dan is $S$ geen functie, want $x_1$ heeft twee "beelden". En $S$ is geen functie, want $y_1$ heeft geen beeld. Toch is $S \circ R$ een functie. 


Oef. 4.1.7 (c,d), p.57-58: welke inclusies zijn geldig voor alle $f, A, B$?  c) is waar denk ik; getekend in Venn diagram en beredeneerd. d) is niet waar denk ik; Nog niet echt grondig bewezen voor de rest. 



Oef. 4.1.11 (d,e,f), p. 58 : (zie a,b,c voorbereiding). Voor e) kan tegenvoorbeeld gevonden worden. $A_1=\{ 1 \}, A_2=\{ 2 \} $, en $f = \{ (1,1), (2,1) \}  $. 
Voor d) kan bewijs gevonden worden. Denk ik. Was verwarrend, want assistent heeft a/b/c van oefening omgewisseld met d/e/f. Ene is met unie, andere met doorsnede. 


Oef. 4.1.12 (a,b,c) p. 58: geskipt. 

Oef. 4.2.4 (p.63):  
Gegeven $f$ en $g$ functies, resp. $X \rightarrow Y$ en $Y \rightarrow Z$. 
a) Als f en g injectief, dan $g \circ f$ injectief. 
Injectief = geen dubbels aan $Y$-kant. M.a.w. als twee $x$ waarden verschillend, dan ook $y$ waarden verschillend. Of contrapositie van de gewone definitie: als twee $y$ waarden gelijk, dan ook $x$ waarden gelijk. 
Bewijs: beschouwen we $g(f(x_1))  = g(f(x_2)$ voor willekeurige $x_1, x_2 \in X$. T.B. $x_1=x_2$. 
Voor de buitenste functie $g$, die injectief is, kunnen we direct afleiden dat hun argumenten gelijk moeten zijn, dus $f(x_1)  = f(x_2$. Maar ook $f$ is injectief, dus volgt direct $x_1=x_2$. QED.

b) $f$ en $g$ surjectief. Recap: geen lege $y$-waarden. We willen dus bewijzen: $\forall z: \exists x: g(f(x)) = z$. Nemen we willekeurige $z$. Omdat $g$ surjectief is: er bestaat een $y \in Y$ zodat $g(y)=z$. Maar tevens geldt voor alle $y$ dat er een $x$ bestaat zodat $y=f(x)$, en na invullen: $g(f(x))=z$. Dus een $x$ bestaat steeds voor willekeurige $z$. QED. 



Oef. 4.2.7: 

Oef. 4.2.10: 

Oef. 4.2.13: 


\subsection{Extra oefeningen}

 4.1.8, 4.1.10, 4.2.8, 4.2.12, 4.2.14,

3.1.2 (b,d), 3.1.3 (a), 3.1.8, 3.2.1, 3.2.3 (a), 3.2.9


\section{Oefenzitting 5, 8 nov, H5, wk7}

\subsection{Oefening 5.1.1 p.70 (a)} 

a) Opgave: bewijs dat er geen surjectieve functie $f: \mathbb{E}_n \rightarrow \mathbb{E}_m$ bestaat met $n,m \in \mathbb{N}, n <m$. 

Ter herinnering: surjectie is: geen lege elementen in $Y$, ofte: voor alle $y$ is er een $x$ zodat $f(x)=y$. 

Hint bij oefening was: via inductie op $m$ of $n$. Maar kan ook via Lemma 5.1.8. op pagina 69. Uit ongerijmde: stel er bestaat een surjectieve $f: \mathbb{E}_n \rightarrow \mathbb{E}_m$ met $n<m$. Dan volgt uit Lemma 5.1.8: $|\mathbb{E}_n | \geq | \mathbb{E}_m |$. Dus is $n \geq m$. Dit is een contradictie met het gegeven, QED. 

Via inductie: nog niet gevonden. Niet zo simpel. Thomas sprak van dubbele inductie (zowel op $n$ als op $m$). 



\subsection{Oefening 5.1.2 p.71}

Gegeven: $X$ eindig, en $X,Y$ equipotent. 
Dus er bestaat een bijectie $f:X \rightarrow Y$, maar ook een bijectie $g:Y \rightarrow X$. Omdat $X$ eindig is, bestaat er een bijectie $h:X \rightarrow \mathbb{E}_n$ met $n \in \mathbb{N}$. 
Beschouw de samengestelde functie $h \circ g:Y \rightarrow \mathbb{E}_n$. Deze functie is ook een bijectie (samenstelling 2 bijecties), dus bijgevolg is $Y$ ook eindig. 

\subsection{Oefening 5.1.8 p.71}

Karolien had oplossing gevonden in google drive. Ze defini\"eren een functie $F(A)$ voor elke $A \subset X$ of dus $A \in P(X)$, die $A$ afbeeldt op een natuurlijk getal dat overeenkomt met de binaire voorstelling van elementen die al of niet in $A$ zitten. Dit is die bijectie tussen $P(X)$ en $\mathbb{E}_n$. (nog 1 erbij tellen om te beginnen met 1 in plaats van 0).  

\subsection{Oefening 5.3.4 p.78}



\subsection{Oefening 5.4.1 p.82}


\subsection{Oefening 5.1.3 p. 71} 
Op bord gestaan. 
Gegeven: $X$ is aftelbaar oneindig, en $Y$ is eindig. Te bewijzen: $X \cup Y$ is aftelbaar oneindig. 

Even erover nadenken: 
Omdat $X$ aftelbaar oneindig is, bestaat er een bijectie $f: \mathbb{N}_0 \rightarrow X$, en kunnen we schrijven: $X=\{ x_1, x_2, ... \}$. Voor $Y$ kunnen we schrijven: $Y=\{ y_1, y_2, ... y_n \}$. 
Als we echter $X \cup Y$ bekijken, kunnen we niet zomaar schrijven: $X \cup Y = \{ y_1, y_2, ... y_n, x_1, x_2, ...\} $, want er kunnen $i,j$ bestaan zodat $y_i = x_j$, dus dubbele elementen in de opsomming. 
We moeten dus schrappen, en dan pas tellen. 

Bewijs: 
Beschouw $f': \mathbb{N}_0 \rightarrow (X \setminus Y)$. Ook is $X \cup Y = Y \cup (X \setminus Y)$. Beschouw twee bijecties: $f':  \mathbb{N}_0 \rightarrow (X \setminus Y)$ en $g: \mathbb{E}_n \rightarrow Y$. Defineer $h: \mathbb{N}_0 \rightarrow X \cup Y$ als volgt: 

\[ 
h(k)= \left\{ 
\begin{aligned}
g(k) , 1 \leq k \leq n \\
f(k-n) , k > n
\end{aligned} 
\right.\]

Claim: $h$ in injectief en surjectief. (zie eerdere redenering) 

Noot: bewijs vooral overgeschreven, niet zelf redenering gevolgd. 


\subsection{Oefening 5.1.4}

\subsection{Oefening 5.3.2 p.78}
Gegeven: $X$ eindig, $f:X \rightarrow X$ is een surjectie. Te bewijzen: $f$ is een bijectie. 

Ik denk dat ik volgende van bord heb overgeschreven: 
Stel niet injectief, dan $\exists x \in X: | f^{-1}\{x\}| > 1$. 
Definieer $\Tilde{f}: X \setminus f^{-1}(\{x\}) \to X \setminus \{ x \} : y \mapsto f(y)$. 
Dan is $\Tilde{f}$ een surjectie. En dus is $| X \setminus\{ x\} | \leq | X \setminus f^{-1}(\{ x \}) |$. 
Dus $| X| -1 \leq |X| - | f^{-1}\{ x\} |$
En dus : $|X| -1 < |X| -1$. Contradictie. 

\subsection{Oefening 5.4.2}

\subsection{Oefening 5.4.3}

Als er tijd over is
Oefening 5.4.4
Oefening 5.53 



Extra oefenmateriaal: 5.5.1, 5.5.2, 5.5.4, 5.5.6


\section*{Oefenzitting 6, 15 nov, H7, wk8}
Oefeningen over hoofdstuk 7, week 8. 

Voorbereidingsoefeningen voor thuis.

\subsection*{Oefening 7.1.1 p.104}
Verzameling W van alle woorden. 
\begin{itemize}
    \item [(a)] (x,y) met alle letters van x komen ook voor in y; reflexief, transitief, niet anti-symm (anagrammen bvb. tennis/intens); ook niet symm. ; GEEN orderelatie. 
    \item [(b)] (x,y) letters van x komen achter elkaar en in dezelfde volgorde voor in y; reflexief; transitief; antisymmetrisch; WEL orderelatie; partieel. 
    \item [(c)] (x,y) als y na x komt in alfabetisch-lexicografische ordening; niet reflexief; GEEN orderelatie. 
\end{itemize}


\subsection*{Oefening 7.1.9 p.105}
D: 'deler van' relatie op $\mathbb{N}_0$. Bvb. (1,5), (2,6), (5,100), ... 
Bewijs dat voor $n,m \in \mathbb{N}_0$ geldt dat $sup \{n,m\} = kgv(n,m)$. 
\begin{proof}
Noem $s = sup \{n,m\}$. Dan geldt volgens definitie: 
\begin{itemize}
    \item $\forall x \in \{n,m\}: x \text{ deler van } s$ ($s$ is bovengrens), en 
    \item $s$ is deler van elke mogelijke andere bovengrens.
\end{itemize}
Dus $n$ is een deler van $s$ en $m$ is een deler van $s$. Dus $s$ is tegelijk een veelvoud van $n$ en van $m$. 
Nemen we een willekeurige andere bovengrens $s' \neq s$. Dan geldt, omdat $s$ het supremum is: 
\begin{itemize}
    \item $s'$ veelvoud van $n$ en van $m$
    \item $s$ is een deler van $s'$, of: $s'$ is een veelvoud van $s$ (en $s' \neq s$) 
\end{itemize}
Dan volgt: $s' = s + s + ... s$. Bijgevolg is $s$ kleiner dan $s'$, is en $s$ het kgv. 
\end{proof}

Deel 2 infimum: ik vermoed dat bewijs gelijkaardig verloopt, voorlopig geskipt. 

\subsection*{Oefening 7.3.1 p.111}

\begin{itemize}
    \item [(a)] niet waar; bvb. $-\sqrt{2}$ en $\sqrt{2}$ ; som is 0
    \item [(b)] waar; bewijs uit ongerijmde; stel $x+y=\frac{p}{q}$ dan $qx + qy = p$ en $y=\frac{p}{q}-x$; dus $y$ rationaal; tegenspraak
    \item [(c)] niet waar; bvb. $x=y=\sqrt{2}$ 
    \item [(d)] waar; bewijs uit ongerijmde; stel $xy=\frac{p}{q}$, dan $y=\frac{p}{qx}$ rationaal; tegenspraak.
\end{itemize}

\subsection*{Oefening 7.3.3 p.111}
Bewijs Stelling 7.3.2 geval (2). (stelling op p. 110-111, onvolledigheid van $\mathbb{Q}$). 

Thuis op zitten kauwen. Ondertussen video v/d les gezien. Ik snap het, maar formeel opschrijven moet ik in oefenen. TODO. 

In oefenzitting met Thomas: je kan ook met limieten berekenen. TODO: ook zo eens proberen. 

(Hieronder: oefeningen voor de oefenzitting)
\subsection*{Oefening 7.1.2 p.104}
\begin{itemize}
    \item [(a)] Ja, orderel. en totaaal
    \item [(b)] Nee, niet anti-symmetrisch
    \item [(c)] Ja, en totaal. (mijn oplossing: FOUT!!)
\end{itemize}

Oplossing: ik heb een fout. (c) is niet totaal. Bvb. $(2,-2)$: niet gelijk en ook geen verschillende absolute waarde!! 

\subsection*{Oefening 7.1.6 p. 105} 
Bewijs dat een supremum van $A \subset X$ ($X$ geordend) uniek is (als het bestaat). 
Stel we hebben twee suprema: $s_1$ en $s_2$. Beide zijn een bovengrens, dus volgt uit de definitie dat $s_2 \leq s_1$, maar ook $s_1 \leq s_2$. $X$ is geordend dus relatie is antisymmetrisch, hieruit volgt $s_1 = s_2$. 


\subsection*{Oefening 7.2.1 p. 108} 
Gegeven: 
\begin{itemize}
    \item $x,y \in \mathbb{R}$
    \item $0<x<y$
    \item rekenkundig gemiddelde $g_r = \frac{1}{2}(x+y)$
    \item meetkundig gemiddelde $g_m = \sqrt{xy}$
    \item harmonisch gemiddelde $g_h = \frac{2}{\frac{1}{x}+\frac{1}{y}}$
\end{itemize}

TB 1: $g_r < y$
Bewijs 1: als $x<y$ dan $\frac{x}{2} < \frac{y}{2}$, en dus $\frac{x}{2} +\frac{y}{2}< y$. QED. 

TB 2: $g_m < g_r$
Bewijs: de truuk is: omgekeerd maken met als en slechts als ertussen. 
$\sqrt{xy}<\frac{x+y}{2}$ asa $0 < x^2 - 2xy + y^2$ asa $0 < (y-x)^2$. 
Dan opkuisen en omdraaien: $y-x>0$ dus $(y-x)^2>0$, dus ...

Andere delen van de oefening voorlopig geskipt. 


\subsection*{Oefening 7.2.2 p.109}  
a) opgelost: hier heel kort, vertrekkend van $0<i$: $i<-1$ en $i<0$, maar ook: $-i<0$. Dus $0<0$, tegenspraak want dan $0 \neq 0$. 

\subsection*{Oefening 7.3.4 p.111-112} 
opgelet: a en b niet interessant (vooral prullen), maak eerst c. 

Deel c gemaakt op papier: ook uit ongerijmde, stel supremum $b$, ook met gevalsonderscheiding, stel eerst $b^2>2$, dan volgt uit $a$ blabla etc. etc. Doenbaar, had op papier staan.  

\subsection*{Oefening 7.3.5 p.112}

a) oplossing: $max (A)=1, met 1\in A$, $sup( A)=1$, $min(A)$ bestaat niet (bewijs uit ongerijmde, bvb. via $\frac{1}{2}$ van zogenaamde minimum); $inf(A)=0$. Eerst aantonen dat $0$ ondergrens is. OK. $inf(A) \leq 0$. Dan stellen dat infimum niet $0$ is, stel $inf(A)>0 $, schrijven als breuk, dan tegenspraak te vinden. 
Interessant inzicht: als infimum bestaat en niet in verzameling zit, dan is er geen minimum. Ook bovenaan: als supremum bestaat en niet in verzameling zit, dan is er geen maximum. Want ze zijn eigenlijk hetzelfde qua definitie. 

Extra oefenmateriaal 


\subsection*{Oefening 7.1.3}
\subsection*{Oefening 7.1.10}
\subsection*{Oefening 7.1.12}
\subsection*{Oefening 7.4.2 }


\section{Oefenzitting 7, 22 nov, H8, wk9}

Assistent: Tey (Berendschot). 


Voorbereidingsoefeningen voor thuis: 

\subsection{Oefening 8.1.1, p.120}
Opgave: bewijs gevolg 8.1.5 (op p.117). 

Gevolg 8.1.5: 
\begin{itemize}
    \item [(a)] $a,b \in \mathbb{R}, a>0$, dan is er een $n \in \mathbb{N}_0 $ met $an>b$. 
    \item [(b)] $a \in \mathbb{R}, a>0 $, dan is er een $n \in \mathbb{N}_0 $ met $1/n < a$. 
\end{itemize}

Dit is gevolg van propositie 8.1.4 (Archimedische eigenschap re\"ele getallen): voor elke $x \in \mathbb{R}$ is er een $n \in \mathbb{N}$ met $x<n$. 

Bewijs voor (a): Neem $\frac{b}{a}$, wat ook in $\mathbb{R}$ zit. Wegens Archimedische eigenschap  re\"ele getallen: er bestaat een $n \in \mathbb{N}$ met $\frac{b}{a}<n$, en dus $b<an$ ($a>0$ dus orde blijft bewaard na vermenigvuldiging met $a$). 


\subsection{Oefening 8.1.3, p.120}
\begin{itemize}
    \item [(a)] $A= [ a,b ] $: $sup (A)=b$, $max (A)=b$. 
    \item [(b)] $A= ] a,b [ $: $sup (A)=b$, $max (A)$ bestaat niet. 
\end{itemize}



\subsection{Oefening 8.1.7, p.120}
Gegeven supremumeigenschap op totaal geordend veld $(\mathbb{F},+,\cdot, \leq)$. Bewijs dat ook infimumeigenschap geldt. 
Oops, ik dacht dat ik dit rap kon bewijzen, maar ben gewoon de mist in gegaan, helemaal verkeerd en te eenvoudig opgevat/te kort door de bocht gegaan! 
Ten eerste: supremumeigenschap is niet het bestaan van een supremum in $F$, maar wel: elke niet-lege naar boven begrensde deelverzameling heeft een supremum. 
Dus we moeten vertrekken van een willekeurige naar boven begrensde verzameling $X \subset F$. Deze heeft dan per gegeven een supremum. 
Ik denk dat we dan moeten kijken naar $\{ -x | x \in X  \}$. In deze verzameling is de orde omgekeerd als we overeenkomstige $x$ bekijken, deze verzameling is naar onder begrensd, en we kunnen alles omdraaien. Hiervoor gaat het waarschijnlijk wel allemaal kloppen. 



\subsection{Oefening 8.3.1, p.124}  
Bewijs dat vgl (8.10) een equivalentierelatie op $\mathbb{N} \times \mathbb{N}$. Vergelijking (8.10) is: \[
(n,m) \sim (p,q) \text{ als en slecht als } n+q=m+p
\]

T.B. : $\sim$ is reflexief, symmetrisch en transitief. 

voorlopig geskipt, lijkt me niet super moeilijk. 



Oefeningen voor de oefenzitting 

\subsection{8.1.2, p.120}
Zij $a<b$. Laat zien dat interval $] a,b[ $ oneindig veel rationale getallen bevat, en ook oneindig veel irrationale getallen. 
Nogal lang op zitten zoeken. Eerste idee: via bijectie f van $] a,b[ $ naar $\mathbb{R}$. Ander idee: via propositie 8.1.8, dat $\mathbb{Q}$ dicht ligt in $\mathbb{R}$ (tussen elke 2 re\"ele getallen ligt nog een rationaal getal). 

Gevonden met tips van Tey: 
\begin{itemize}
    \item kan je er 1 vinden? 
    \item en kan je er een tweede vinden? 
\end{itemize}

Bewijs voor (a) oneindig veel rationale getallen: 
Nemen we twee re\"ele getallen $x,y \in ] a,b [, x<y$.  Kan altijd, bijvoorbeeld rekenkundig gemiddelde $y = g_r = \frac{a+b}{2}$ en meetkundig gemiddelde $x = g_m = \sqrt{a b} $ van de grenzen van het interval. We hebben dan $a < x < y <b$ met $x,y \in ] a,b [$. 
Aangezien  $\mathbb{Q}$ dicht ligt in $\mathbb{R}$: $\exists q_1 \in \mathbb{Q}: x < q_1 < y$. 
Via dezelfde eigenschap toegepast op $q_1$ en $y$, vinden we ook een $q_2$ zodat:  $x < q_1 < q_2 < y$. Hierbij liggen $q_1$ en $q_2$ duidelijk in het interval.
We kunnen zo oneindig doorgaan, waarmee bewezen is dat er oneindig veel rationale getallen in het interval liggen. 

Redenering voor irrationale getallen: hoe vind ik het eerste? Ik weet bijvoorbeeld dat $ 0< \frac{\sqrt{2}}{2} < 1$ irrationaal is. Ik weet ook dat bijgevolg $ 0 < (q_2-q_1)\frac{\sqrt{2}}{2}   < q_2 - q_1 $ als $q_1<q_2$, en dit getal nog steeds irrationaal is (product van rationaal en irrationaal is irrationaal). Dus ook: $ q_1  < q_1+ (q_2-q_1)\frac{\sqrt{2}}{2}   < q_2 $ is irrationaal. 

Bewijs voor (b) oneindig veel irrationale getallen: we vinden via vorige bewijs altijd twee rationale getallen zodat $a < q_1 < q_2 < b$. Nemen we dan $x_{i1} = q_1+ (q_2-q_1)\frac{\sqrt{2}}{2}$. Dan is $x_{i1}$ irrationaal, en dan geldt $a < q_1 < x_{i1} < q_2 < b$. Vanaf nu kunnen we oneindig veel nieuwe irrationale getallen $x_{i2}, x_{i3}, ... $ vinden door te stellen $x_{i(j+1)} = \frac{x_{ij}+q_2}{2} $, met andere woorden het rekenkundig gemiddelde tussen laatst gevonden irrationaal getal, en $q_2$. 
QED. 
 
\subsection{8.1.5, p.120}
Dit volgt rechtstreeks uit oefening 8.1.2. 

 
\subsection{8.1.6, p.120}
ivm driehoeksongelijkheid

Ik dacht: eens kijken hoe de driehoeksongelijkheid bewezen is in de cursus. Wat blijkt: bewijs niet in cursus. 
Dus waarom niet eens eerste de gewone driehoeksongelijkheid proberen bewijzen: $| x + y | \leq |x| + |y|$ (vgl. 8.4 in cursus). 
Wat eigenschappen opgeschreven, wat zitten nadenken, maar niks elegant gevonden. Alleen via gevalsonderscheid die ook definitie van absolute waarde zit, geraak ik er. En blijkt dat ongelijkheid alleen geldt als $x$ en $y$ tegengesteld teken hebben. 



\subsection{8.2.3, p.121}
Vraag gesteld: notatie $:=$, betekent die "per definitie"? Tey zei van wel, maar zei ook: gebruik dat zelf maar niet, en zet altijd in woorden er bij als iets een definitie is. 
Oefening niet meer afgemaakt. 

\subsection{8.2.4, p.121}
 
\subsection{8.2.6, p.122}

Als er tijd over is  8.3.4, 8.4.1 (geen examenstof) 

\section{Oefenzitting 8, 29 nov, H9, wk10}
(wk 10 niet gegaan) 

Voorbereidingsoefeningen voor thuis.


\subsection{9.2.2 (a) p.136}

Tijdens maken oefening, vroeg ik me af: mag ik worteltrekken over een $<$ of $\leq$? 
Voor positieve re\"ele ongelijkheden lijkt me dit geen probleem. 

Gegeven: $(a_n)$ met $a_n = \frac{2}{\sqrt{n}}$, $n \in \mathbb{N}_0$

T.B. \[
\lim_{n \to \infty} a_n = 0 \] 

Bewijs: 
Neem $\epsilon > 0$ willekeurig. En neem dan $n_0 > \frac{4}{\epsilon^2}$. Dan geldt voor alle $n \geq n_0$: $n > \frac{4}{\epsilon^2}$. 
todo: vervolg...



\subsection{9.2.5}

\subsection{9.3.1}


\subsection{9.4.1} 



Oefeningen voor de oefenzitting 

\subsection{9.2.2 (b)}

\subsection{9.2.4}

\subsection{9.2.6 (b)}

\subsection{9.3.3 (a)}

\subsection{9.4.4 (tweede limiet)}

\subsection{9.4.5 }

Als er tijd over is 

\subsection{9.3.4}

\subsection{9.5.1}


\section{Oefeningen over H10, wk11}

Theorie: 
Rij heeft begin maar geen einde. Functie van N naar R, begint algemeen op 0 dus. a0.
Begrensd vs convergent: 
Rij die is als sinus : begrensd maar niet convergent. 
Rij die naar oneindig gaat als n naar oneindig gaat: 
Niet convergent en niet begrensd. 
Stelling die zegt: als convergent, dan begrensd. 
Dus maar drie opties qua combinatie begrensd/convergent. : 
1. Begrensd convergent
2. Begrensd niet convergent : dus limiet bestaat niet. 
3. Onbegrensd 

Dan de lim sup / lim inf : die bestaat wel altijd voor begrensde rijen, en als limiet bestaat zijn lim sup en lim inf er beide aan gelijk. 

Voorbereidingsoefeningen voor thuis.

10.1.1 (a), 10.2.1, 10.2.4 (a) 



Oefeningen voor de oefenzitting 

 10.1.1 (b), 10.1.4, 10.1.6, 10.2.2, 10.2.3 (a) 10.2.5 (a), 10.3.3 

 Als er tijd over is 

10.1.2, 10.2.7, 10.3.8 

\section{Oefeningen over H11, wk12} 


Voorbereidingsoefeningen voor thuis

11.1.1 (a), 11.1.3, 11.2.1

Oefeningen voor de oefenzitting 

11.1 (b), 11.1.2, 11.2.2, 11.2.3, 11.2.4

Als er tijd over is 11.3.1


\section*{NEXT}

\end{document}