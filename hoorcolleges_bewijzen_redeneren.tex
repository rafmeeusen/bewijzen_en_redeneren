\documentclass{article}
\usepackage{graphicx} % Required for inserting images
\usepackage{amsmath} 
\usepackage{amsfonts} % moet erbij om verzamelingen N,R... netjes te krijgen

% mark paragraphs with empty line instead of indented first line
\setlength{\parindent}{0em}
\setlength{\parskip}{1em}

\title{Nota's hoorcolleges bewijzen en redeneren}
\author{Raf Meeusen}
\date{2023-2024}

\begin{document}

\maketitle

\section{Les 1, 27 sept}

Termen: \begin{itemize}
    \item proposities
    \item predicaten
\end{itemize}


voorbeeld predicaat: $x^2 \geq 0 $
(vrije variabele erin; waar of onwaar pas te bepalen als meer over x geweten is; bvb. reeel of complex) 

Een predicaat wordt een propositie als de vrije variabelen gespecifieerd zijn. 
Bvb.: voor alle reele getallen $x$ geldt dat $x^2 \geq 0$  ; dit is wel een propositie! 

Nog termen: 
\begin{itemize}
    \item beweringen
    \item connectief/ connectieven / logische connectieven (of , en, als ... dan etc.) 
    \item waarheidstabel (op te stellen voor elk connectief) 
\item implicaties (dit is de "als ... dan"; geen simpele) 
\item equivalenties ("als en slecht als" / asa / $\Leftrightarrow$) 

\end{itemize}


Implicaties: 
hoe begrijpen? als P waar is, dan \emph{moet} Q ook waar zijn; (dus als P niet waar is, kunnen we niks over Q zeggen) 
Voorbeeld: als $\pi < 3$, dan is 1+1=100 ; deze implicatie is "WAAR" !! 
Ook onthouden: niks te maken met causaal verband !! 
Voorbeeld: als 1+1=2, dan is $\pi$ irrationaal ; deze implicatie is "WAAR" !! 

Equivalentie: zelfde als twee implicaties die geldig zijn ($P=>Q$ en $Q=>P$). 
$P <=> Q $ betekent zelfde als : $(P=>Q)$  EN  $(Q=>P) $

Raadsel. 3 goden, allen oneindig slim. 1 god geeft systematisch juiste antwoord, 1 god geeft systematisch foute antwoord, en derde god geeft willekeurig antwoord. 
Godentaal: da / na ; maar we weten niet of da ja is of nee. 
We mogen 3 vragen stellen aan die goden, en bepalen welke god welke waarheid spreekt. 

"Soorten" bewijzen: 
\begin{itemize}
    \item rechtstreeks bewijs (cursus: direct bewijs) 
    \item gevalsonderscheid
\end{itemize}


Stappen in een bewijs: NIET met $=>$ aangeven. 
onderscheid: 
symbool $=>$ is een logisch connectief 
de stappen in een bewijs zijn een "verhaal". 
(analogie: zinsdeel , zin, verhaal) 

Achterwaarts construeren: 
eerder een techniek om een bewijs te vinden; niet zozeer een soort bewijs. 

Als je bewijs leest, denk je: hoe hebben ze dit ooit gevonden. Wel, misschien hebben ze eerst achterwaarts geconstrueerd, maar die redenering niet mee opgeschreven. Het opgeschreven bewijs is dan misschien een direct bewijs, maar het kan lijken op "hoe hebben ze dit gevonden"? "dit komt uit het niets". 

Nog soorten: 
\begin{itemize}
    \item Bewijs uit het ongerijmde 
\item Bewijs met contrapositie
\item Bewijs met volledige inductie
\end{itemize}

(volledige inductie: typisch voor uitspraken over de natuurlijke getallen)
Opmerking: natuurlijke getallen bevatten 0 (niet zo in alle landen!) 

Inductie stappen: basisstap, inductiestap (met inductiehypothese: "stelling is geldig voor n")

Stelling: formule voor n-de getal uit rij van Fibonacci: $a_n = 1/\sqrt{5} * (...) $, voor alle natuurlijke getallen n. 
Bewijs in cursus met volledige inductie. 

\section{Les 2, 4 okt}

Hoofdstuk 2. Verzamelingen en kwantoren. Betekenis element ($x \in X$). Bekende verzamelingen $\mathbb{N}, \mathbb{Z}, \mathbb{Q}, \mathbb{R}, \mathbb{C}, \mathbb{N}_0, ... $
Deelverzameling $A \subset B$. Onze afspraak: deelverzameling mag ook gelijke verzameling zijn (niet bij iedereen de conventie, er bestaat ook een symbool $\subseteq$, maar laat ons dat niet gebruiken). De lege verzameling $\emptyset$. Opsomming elementen $\{a,b,c\}$. Via beperkende eigenschap: $\{x\in \mathbb{R} \mid x>2 \} $. Of een formule: $\{x^5 -3x  \mid  x \in \mathbb{Z} \} $

Bewerkingen met verzamelingen: $A \cup B$, $A \cap B$. Universum U, en complement-verzameling $A^C = U \setminus A$. Distributiviteit van doorsnede op unie. 

Eigenschap: complement van een doorsnede is gelijk aan unie van complementen: $(A \cap B)^C = A^C \cup B^C$. Kan je aantonen met Venn-diagram, en de vier mogelijkheden beschouwen (in A en in B, in A en niet in B, ...). Maar kan ook anders. Typische mogelijkheid voor gelijkheid van twee verzamelingen $A en B$, is in twee stappen. Stap 1: bewijs dat $A \subset B$, en stap 2: bewijs dat $B \subset A$. 

Voorbeeldbewijs in les gezien van $(A \cap B)^C = A^C \cup B^C$. Principe voor eerste inclusie: neem willekeurige $x$ uit $(A \cap B)^C$. Dan zit  $x$ niet in $A \cap B$. En dus geldt: $(x\notin A)$ of $(x\notin B)$. Dus geldt: $(x \in A^C)$ of $(x\in B^C)$. En daarom ... En dan nog tweede inclusie bewijzen. En klaar. 

De machtsverzameling van verzameling $X$, genoteerd als $P(X)$, verzameling van alle deelverzamelingen. Noot: wij noteerden dat altijd $2^X$. Tricky: $P(\emptyset) = \{ \emptyset \}$, dus niet gewoon gelijk aan de lege verzameling. 

Cartesisch product van twee verzamelingen: $X \times Y$. 

Algemene unies en doorsnedes: unie of doorsnede gebaseerd op verzamelingen met een index. 

Kwantoren: "voor elke": $\forall$  en "er bestaat" $\exists$. Men noemt ze respectievelijk  de universele en de existenti\"ele kwantor. Dan in les even op andere manier geschreven (verzameling gelijk aan A voor de universele, en verzameling gelijk aan lege verzameling voor de existenti\"ele). Dat blijkt wel handig om bvb. volgende uitspraak als waar of onwaar te bekijken: $\forall x \in \emptyset : x>2$. Die is niet evident, en die is waar, want als je het herschrijft in die andere vorm, klopt het helemaal. 
Daarna ook nog andere uitdrukking met implicatie bekeken. Mmm, opnieuw implicaties, blijft lastig om over na te denken. Als linkerkant niet waar is, is de implicatie altijd waar. (maar daarom niet de rechterkant!)... 

Combinaties kwantoren. Volgorde is van belang. 

Negatie ($\lnot$) van de $\forall$ wordt een $\exists$. Negatie van de $\exists$ wordt een $\forall$. 


\section{Les 3, 11 okt}

Hoofdstuk 3. Relaties. Definitie als deelverzameling van cartesisch product. Dus een koppel $(x,y)$ kan element zijn van de relatie $R$. Iets formeler kan een relatie ook gedefinieerd worden als drietal $(R,X,Y)$. 

Inverse relatie $R^{-1}$: koppels $(y,x)$ met $(x,y) \in R$. 

Samenstelling van relaties R en S, notatie $S \circ R$. Engels: composition. Relaties \emph{op een verzameling} is de formulering als $X=Y$. Graaf, gerichte graaf: met pijlen de relatie tekenen. 

Equivalentierelaties. (andere: orde-relaties, zie later; nog andere: functies, zie volgende week). Het zijn relaties op X, die reflexief, symmetrisch en transitief zijn. Voorbeeld: "heeft dezelfde verjaardag als" in verzameling studenten. 

Definitie partitie $\mathcal{P} $ van X. Partitie is deelverzameling van de machtsverzameling $P(X)$. Via eigenschappen gedefinieerd. 

Stelling die partities en equivalentierelaties aan mekaar koppelt (1-op-1 relatie ertussen). Equivalentieklassen gedefinieerd tussendoor in de les. Notatie $[x]_R$. Notatie met tilde $\sim$ voor equivalentierelaties en ook bij equivalentieklasse gebaseerd op relatie.  Dan soort bewijs gezien, dat elke equivalentierelatie een partitie definieert, door elke van de 3 eigenschappen na te gaan. 

Het woord "lemma" gebruikt: voor hulp-resultaat, een minder belangrijke stelling. 

Even terug naar eigenschappen van relaties: reflexief, symmetrisch, transitief: opdracht, probeer eens alle combinaties te vinden: reflexief, niet symmetrisch, niet transitief, etc. 


\section{Les 4, 18 okt}
Video gevolgd (van 2021 of zo). 

Hoofdstuk 4. Functies. In deze cursus ingevoerd als speciaal soort relaties. Definitie: voor elke $x$ is er \emph{precies \"e\"en y} zodat $(x,y) \in \mathbb{R}$. 
Notatie: $f: y=f(x), f:X \rightarrow  Y: x \mapsto f(x)$
(todo: hoe zat het met deze notatie? is dat met , of : erin? nakijken in boek) 

Elke element links moet een pijl hebben. $X$ en $Y$ zijn onderdeel van de definities van de functie! Andere $X$ geeft andere functie. 
Definities domein en codomein. Opgelet: het beeld van $X$ is niet per se gelijk aan het codomein. 

Samenstellen van functies. Samengestelde van twee functies is ook een functie. 

Definities eenheidsfunctie. 

Functie-voorschrift: hoeft niet altijd in expliciete formule. 

Definitie "beeld van" een deelverzameling $A \subset X$. En definitie invers beeld van een deelverzameling $B \subset Y$. Opgelet: de notatie $f^{-1}(B)$ is voor inverse beeld, en wil niet zeggen dat de functie $f^{-1} $ bestaat! Verwarrende notatie! Maar het is zo. 

Vraag in theorieles bezien en bewijs ervoor: is $B = f( f^{-1}(B) $? Bewijs via 2 inclusies. Antwoord: niet waar. Slechts een van de twee inclusies is geldig (eerste niet geldig, tweede wel). Alleen voor speciale (surjectieve) functies zijn beide geldig. Tegenvoorbeeld gezien in de les. Best moeilijk om over na te denken. Zie ook oefeningen. 
$f (f^{-1}(B)) \subset B$ is wel waar. Bewijs gezien in de les. 

(pauze)

Dan eigenschappen van functies bekeken: "injectief" , "surjectief". Definities gezien. Tip: in Venn-diagrammen bekijken (aantal pijlen op 1 $y$-waarde). 

Dan naar $tan x$ functie gekeken. Afhankelijk van domein, is de functie wel of niet injectief. Domein kleiner dan een periode: injectief. Noteer: $\pi /2$ en ook andere veelvouden ervan zitten natuurlijk nooit in domein. 

Notatie: nog een verwarrende notatie: $f^{-1}(y)$ kan betekenis hebben van invers beeld, van het singleton $y$. Dus $f^{-1}(y) = f^{-1}(\{y\})$. Hangt van context af, want kan ook inverse functie betekenen ... 

Combinaties injectief, surjectief: interessante oefening: alle combinaties kunnen. En \"e\"en combinatie krijgt een naam: injectief \emph{en } surjectief noemen we bijectief. Surjectief: elke y heeft minstens 1 pijl, injectief: elke y heeft hoogstens 1 pijl, bijectief: elke y heeft exact 1 pijl. In plaats van met pijlen: aantal elementen van invers beeld $f^{-1}(y)$. 

Een bijectieve functie heeft een inverse functie. Symbool $f^{-1}$, niet gebruiken hiervoor als $f$ geen bijectie is. 

Eigenschappen van $f^{-1}$: functie samenstellen met zijn inverse, geeft eenheidsfunctie. Opgelet: ene geval is het $1_X$, in andere geval $1_Y$. Ook omgekeerd: als $f$ en $g$ functies zijn zodanig dat samenstellen in beide volgordes de eenheidsfunctie geeft, dan is $f$ een bijectie en is $g$ de inverse van $f$. We noemen $f$ dan inverteerbaar. Een bijectieve functie is een inverteerbare functie. 

Definitie: een permutatie is een bijectie \emph{op} $X$. (vooral gebruikt voor eindige verzamelingen). 

Begin gemaakt van hoofstuk 5 Kardinaliteit. 

\section{Les 5 voorbereiding}

Kijken naar youtube filmpje (https://www.youtube.com/watch?v=RZL5bKk9IHs): 
\begin{itemize}
    \item aftelbaarheid, gelijke cardinaliteit voor eindige verzamelingen
    \item aftelbaarheid of gelijke cardinaliteit voor oneindige verzamelingen (Galileo Galilei geraakte er niet uit) 
    \item verband cardinaliteit met injecties/surjecties 
    \item verband pigion hole principle (met injecties) 
\end{itemize}

Dan even nagedacht hoe ik dat in godnaam ga onthouden, verschil injectie en surjectie. Bijectie is duidelijk: als het beide is, is het bijectief. 

Eerste ezelsbruggetje: no injection possible from pigeons to holes, if more pigeons than holes. De functie is dan iets dat duiven afbeeldt op hokjes. Als er meer $x$ zijn dan $y$, en uit elke $x$ moet een pijl vertrekken naar een $y$, dan komen er twee pijlen samen om een functie te hebben, en dan is er geen injectie meer, de $x$ "passen niet" in de verzameling $Y$. Opgelet: het is niet omdat qua cardinaliteit een injectie mogelijk is, dat elke functie die je bedenkt ook echt een injectie is! 
Injectie = geen dubbele bij $Y$. Leegstaande y mogen wel. 

Tweede ezelsbruggetje: surjectie = "sur"jectie = "op" = beeldt af "op Y", dat wil zeggen, de volledige Y is "gedekt" (geen y waar geen enkele pijl toekomt; dubbele mogen wel). Met andere woorden: GEEN LEGE bij Y. 

Noteer: we spreken over functies, dus elke $x$ heeft een beeld. Overal vertrekt een pijl, altijd, injectief of niet, surjectief of niet. 

Dan moesten we nog Cantor-Bernstein-Schr\"oder stelling overdenken. En bewijs vinden voor eindige verzamelingen. Stel twee verzamelingen $X$ en $Y$: als er twee injecties $f$ en $g$ bestaan, waarbij $f:X \rightarrow Y$ en $g:Y \rightarrow X$, dan bestaat er ook een bijectie $h: X \rightarrow Y$. 

Poging bewijs eindige verzamelingen: gegeven 2 injecties $f:X \rightarrow Y$ en $g:Y \rightarrow X$. Te bewijzen, er bestaat een bijectie $h: X \rightarrow Y$. In woorden: injectie $f$ bestaat, dus elke $x$ heeft zijn eigen $y$, geen dubbele $y$. Pfff, opgegeven voor het moment, te lastig. Eerst oefeningen doen denk ik. 


\section{Les 5, 25 okt}

(laatste les voor de 3-punters, fysica o.a.)

Proefexamen (tussentijdse toets) maandag aanstaande: er zijn 3 auditoria, zie Toledo. 
Huistaak deze week: indien voor donderdag 14u, dan zal die op vrijdag nagekeken zijn, en heb je feedback voor de toets maandag. 

Hoofdstuk 5: Kardinaliteit. 
Twee verzamelingen $X$ en $Y$ zijn equipotent (gelijkmachtig) als er een bijectie $f: X \rightarrow Y$ bestaat. Men zegt ook wel: $X$ en $Y$ hebben dezelfde kardinaliteit. 

(we weten: bij een bijectie bestaat de inverse functie $f^{-1}$) 

De relatie "equipotent met" is reflexief, symmetrisch en transitief. (bewijs voor transitief: gebaseerd op feit dat $g \circ f$ ook een bijectie is). Dus er bestaan een equivalentieklassen i.v.m. "equipotentie".

Speciale verzamelingen: $ \mathbb{E}_n = {1,2,...,n}$ en $\mathbb{E}_0 = \emptyset$. Als $X$ equipotent is met $\mathbb{E}_n$, dan heeft $X$ $n$ elementen, of: de kardinaliteit is $n$. Notatie $|X| = n$, of $ \# X=n$. $X$ is dan eindig. Zoniet (indien niet eindig), is $X$ oneindig. 

Als je een oneindige verzameling $X$ kan noteren via opsomming met $...$ op het einde, dan is die verzameling equipotent met $\mathbb{N}$ en/of met  $\mathbb{N}_0$. We noemen deze verzamelingen $X$ aftelbaar oneindig. 

Stelling: als $X$ en $Y$ aftelbaar zijn, dan is $X \cup Y$ ook aftelbaar. 
Bewijs (alleen voor de oneindig aftelbare, dus $X$ en $Y$ equipotent met $\mathbb{N}$. 
Bewijs redelijk triviaal: schrijf zowel $X$ als $Y$ in de vorm $\{x_1, x_2, ...\}$ (zie eerder). We kunnen dan de unie ook op zo'n manier schrijven. Opgelet: er zouden dubbels in kunnen zitten als $X$ en $Y$ niet disjunct zijn, dus die dubbels zouden we moeten weglaten. Dus moeilijk om formule voor bijectie op te schrijven, maar we zien in dat het kan. 

Uitbreiding stelling: niet twee aftelbare samenvoegen met unie, maar meerdere (eindig aantal $k$ verzamelingen in een algemene unie gieten). Blijft gelden uiteraard. 

Andere stelling: gegeven verzamelingen $X_i$ (met $i \in I$. Als $I$ aftelbaar is, en elke $X_i$ is aftelbaar, dan is $\bigcup X_i$ ook aftelbaar. 
In woorden: een aftelbare unie van aftelbare verzamelingen is aftelbaar. 
Bewijs in de les. Via truuk zoals bij aftelbaarheid $\mathbb{Q}$. In soort tabel zetten (rijen onder mekaar), en dan zigzaggend. Dus we hebben een techniek voor een aftelling, dus is die algemene unie aftelbaar. 

Dan terug naar $\mathbb{Q}$. Getallen van de vorm $\frac{m}{n}$ met $m \in \mathbb{Z}, n \in \mathbb{Z}_0$. 
Defini\"eren we $X_n = \{  \frac{m}{n} | m \in  \mathbb{Z} \} $, 
dan is $\mathbb{Q} = \bigcup_{n \in \mathbb{N}_0} X_n$. 
We kunnen dan zien dat elke $X_i$ equipotent is met $\mathbb{Z}$, voor $X_1$ geldt zelfs $X_1 = \mathbb{Z}$. (todo: dit wat beter uitleggen, in de les is dat gedaan). Dus omdat alle $X_n$ aftelbaar oneindig zijn, is wegens vorige stelling ook $\mathbb{Q}$ aftelbaar oneindig. 

Wat met $\mathbb{R}$? Niet aftelbaar. Maar bewijs moet nog komen. 

Definitie: men noemt een verzameling die niet aftelbaar is, "overaftelbaar". (overaftelbaar is altijd oneindig). 

Andere stelling: als $X$ een verzameling is, dan heeft machtsverzameling $P(X)$ een grotere kardinaliteit. Met andere woorden: er is een injectie $f: X \leftarrow P(X)$, maar er bestaat geen surjectie $f: X \leftarrow P(X)$. 
Hoe vinden we zo'n injectie? Beeldt elk element $x \in X$ af op zijn singleton in $P(X)$. Er is dus duidelijk een deelverzameling van $P(X)$ die equipotent is met $X$. Maar er zit nog meer in de machtsverzameling. 
Bewijs dat er geen surjectie bestaat: stel $f$ is een willekeurige functie, ... Te bewijzen: $f$ kan niet surjectief zijn. Bewijs: voor de verzameling $A$ van alle $x$ waarvoor ... 
Nota: er kwam een notatie $x \notin f(x)$ voor in dit bewijs, zie p. 70 in cursus. Ik snap deze notatie niet goed. Misschien is dit de notatie "beeld" van singleton $\{x\}$. Achteraf bekeken: moet wel, anders kan je $\in$ symbool niet gebruiken. 

(pauze) 

Taalvaardigheid. Ook deel van de opleiding: teksten schrijven, presentaties maken. Ook op huistaken, examen: geen afkortingen, geen Engelse termen (of toch niet te veel). 

Gezien vorige stelling: $P(\mathbb{N}_0$ is overaftelbaar. 

Stelling: $\mathbb{R}$ is overaftelbaar. Cantor heeft daar over nagedacht. Bewijs: beginnen we te kijken naar $[ 0, 1]$, en alvast bewijzen dat dit ook al overaftelbaar is. Even stilstaan bij "wat zijn re\"ele getallen"? We komen dan snel bij decimale ontwikkeling. $x = 0,a_1 a_2 a_3 ...$ met $a_i \in \{ 0,1,...9\}$. Opgelet: getal $1 = 0,99999...$. Gelijkaardig geldt ook: $0,20000... = 0,19999...$. Dus een decimale ontwikkeling is niet altijd uniek, maar de $00000$ versus $99999...$ is de enige waar meerdere schrijfwijzen bestaan. 
Bewijs uit ongerijmde. Stel $f: \mathbb{N}_0 \rightarrow [ 0,1] $ is bijectief. ... rest bewijs niet genoteerd. (p.80-81 in cursus). Truuk: het diagonaalargument. Kan altijd om de $b_i$ zo te kiezen (altijd nog 9 keuzes over). 
In les ook nog even bekeken dat er geen $,...0000000...$ versus $,...999999....$ probleem in zit. Maar als we nooit een $0$ of een $9$ kiezen voor onze $b_i$, dan doet dat probleem zich nooit voor. In cursus dat dat iets anders geschreven. 
Conclusie: $f$ kan geen bijectie zijn. 


Stelling "Cantor, Bernstein, Schr\"oder" (CBS). Stel $X,Y$ verzamelingen waarvoor geldt dat er een injectieve functie $f: X \rightarrow Y$ is, en ook een injectieve functie $g: Y \rightarrow X$. Dan zijn $X$ en $Y$ equipotent (en bestaat er ook een bijectie). Bewijs niet volledig gezien. Cursus p.74, paragraaf $5.3$. 
Redenering: als $f$ injectief, dan is $X$ equipotent met $f(X) \subset Y$, en is dus $|X| \leq |Y|$. Maar omdat we ook een $g$ hebben: $|Y| \leq |X|$. 
Bewijs is geen examenstof. 

Gebruik van CBS: bewering dat $[ 0,1 ]$ en $\mathbb{R}$ equipotent zijn, kan zo bewezen worden. (twee injecties vinden, en klaar). Terwijl: een bijectie vinden, is veel moeilijker. Dus CBS helpt hier om dit te bewijzen. Injectie van links naar rechts: triviaal (identieke functie). Injectie andere richting: bvb. bgtan. (en wat foefelen). Ik dacht aan "delen door een dichtbijliggend geheel getal". Niet helemaal zeker of dit een injectie is. TODO: hier eens over nadenken.  

TODO: wat is leerstof juist? staat dat ergens opgesomd of moet je les volgen om te weten?? 


\section{Les 6, 8 nov}

Hoofdstuk 6: overgeslagen, niet kennen. Dus ineens hoofdstuk 7: orderelaties. 

Tot nu al twee types relaties gezien: equivalentierelaties, en functies. Nu nog een ander type: orderelaties. Een relatie $R$ op $X$ is een orderelatie als aan drie voorwaarden is voldaan: reflexief, anti-symmetrisch en transitief. Antisymmetrisch wil zeggen: als $(x,y)$ in de relatie $R$ zit, en ook $(y,x)$ in de relatie $R$ zit, dan moet $x=y$. Of in contrapositie: als $x \neq y$ dan $(x,y) \notin R \lor (y,x) \notin R$. 

Andere notatie voor $(x,y) \in  R$ met $R$ een orderelatie: $x \preceq y$. 

Voorbeeld: de inclusie $\subset$ in $P(X)$ is een orderelatie. Opgelet!! Bij orderelatie $\leq$ op $\mathbb{R}$ geldt: als $x \neq y$ dan $x < y$ of $y < x$. Maar bij $\subset$ geldt volgende NIET: als $A \neq B$ dan $A \subset B$ of $B \subset A$. TODO: eens over nadenken, want contrapositie blijft toch universeel geldig... snap die uitleg niet goed. Zijn uitleg is in het kader van totale ordening. 

Definitie: orderelatie is totaal als ... (elke 2 elementen zitten in ene of andere richting in relatie). Men spreekt van totaal geordende verzameling. 

Definitie "het maximum", "het minimum". Opgelet, in cursus in andere volgorde uitgelegd. 

Voorbeeld: $(p,q)$ in $\mathbb{N}_0$ met $p$ is een deler van $q$. Dit is een parti\"ele ordening. Voorbeeld: $A=\{3,4,6\}$ met "is een deler van". Bestaat het maximum? Niet duidelijk uit definitie. Antwoord: geen maximum. 

Definitie maximaal element: betekent dat er geen groter bestaat in de verzameling. 
$A=\{3,4,6\}$ met "is een deler van" heeft dan twee maximale elementen: $4$ en $6$. 
Bovengrens van $A$: zit niet per se in $A \subset X$, maar in $X$. Dan ook: supremum definitie. Ondergrens, infimum. 

(pauze)

deel 2 gebrost. 

\section{... veel gebrost}


\section{Theorie H9}

Eigen samenvatting: 
Rij heeft begin maar geen einde. Functie van $\mathbb{N}$  naar $\mathbb{R}$, begint algemeen op 0 dus ($a_0$).
Begrensd vs convergent: 
Rij die is als sinus : begrensd maar niet convergent. 
Rij die naar oneindig gaat als n naar oneindig gaat: niet convergent en niet begrensd. 
Stelling die zegt: als convergent, dan begrensd. 

Dus maar drie opties qua combinatie begrensd/convergent : 
\begin{itemize}
    \item [1.] Convergent (en bijgevolg begrensd, propositie p.138) 
    \item [2.] Begrensd divergent (=begrensd en niet-convergent) : dus limiet bestaat niet. 
    \item [3.] Onbegrensd (en bijgevolg divergent) 
\end{itemize}

Interessante voorbeelden rijen: 

\begin{itemize}
    \item $a_n=\frac{1}{n+1}$ : convergent (limiet 0) 
    \item $a_n=\sin(n)$ : begrensd, divergent
    \item $a_n= (-1)^n$ : begrensd, divergent
    \item $a_n = \sum_{k=1}^{n} \frac{1}{k} , n \in \mathbb{N}_0$ : divergent ; ($1, 1+\frac{1}{2},1+\frac{1}{2} +\frac{1}{3}, ...$ )
    \item $a_n = \sin(\ln n)$ : begrensd, divergent
    \item $a_n = \sin(\frac{1}{n+1})$ : convergent 
    \item $a_n = \sum_{k=1}^{n} \frac{1}{k^2} , n \in \mathbb{N}_0$ : convergent ; ($1, 1+\frac{1}{4},1+\frac{1}{4} +\frac{1}{9}, ...$ )
\end{itemize}

Opmerking: p.158 in boek niet kennen, maar OOK interessant. Divergentie kan onbegrensd naar $+\infty$, maar kan ook onbegrensd alternerend ($1,-1,2,-2,3,-3,4,-4,...$)

Een rij kan geen $\infty$ bevatten! Dus onbegrensd wil zeggen dat er in de limiet voor zeer grote $n$ ergens iets naar oneindig gaat. Kan niet voor eindige $n$, want voor elke eindige $n$ is er een deelverzameling van $\mathbb{R}$, en die heeft een maximum, dus is er een bovengrens, etc. 

Hoorcollege: uitleg gezien met integraal om te kijken of rij divergeert (rij als som van rechthoekjes, en integraal als boven- of ondergrens of zo). 

Oefening 11.1.1 was ook zeer leerzaam om beter te snappen welke soort rijen er zijn, komt pas paar hoofdstukken later, maar helpt om inzicht te krijgen in types van rijen en convergentie. 

\section{Theorie H11 Cauchyrijen, deelrijen}

Eigen samenvatting: 

Cauchyrij: afstand tussen $a_n$ en $a_m$ is willekeurig klein vanaf ($n,m>n_0$) gekozen $n_0$. Afstand gedefinieerd als $| a_n-a_m|$. 

Twee stellingen: 
\begin{itemize}
    \item als rij convergent is, dan is het een Cauchy-rij
    \item als rij Cachy-rij is, dan is ze convergent 
\end{itemize}

Tiens, wat is dan het verschil? Wel, het is een definitie zonder limiet in, kan een voordeel zijn. En een Cauchyrij kan in elke ruimte gedefinieerd worden waar een afstand bestaat (bvb. $\mathbb{C}$, $\mathbb{R}^n$, ...). Ordening niet nodig, wel afstand. 

Ophopingspunt: had het gelezen, maar eigenlijk pas beter begrepen na tip uit hoorcollege-video: bekijk het voor niet-convergente rij. En maak het aanschouwelijk op een $\mathbb{R}$-$\mathbb{N}$ plot van een rij. 

Toch wat hoorcollege video zitten bekijken. 
Begint met intro over "favoriete limieten". Paar interessante voorbeelden van limieten: 
\begin{itemize}
    \item Hoe snel groeit $n!$ als $n \to \infty$? Divergent natuurlijk, maar met formule van Stirling $s(n)$ is er benadering voor grote $n$ waardoor quoti\"ent $\lim_{n \to \infty} \frac{n!}{s(n)} = 1$. $s(n)$ makkelijk te vinden op wikipedia. 
    \item $\pi(n)$: aantal priemgetallen $\leq n$. Dit divergeert ook natuurlijk, maar ook hier benadering waardoor $\lim_{n \to \infty} \frac{\pi(n)}{n/\ln n} = 1 $. Hadamard/de La Vall\"ee Poussin. 
    \item Centrale limietstelling uit statistiek, in verband met som van oneindig veel stochastische veranderlijken
    \item Affiche van Vlaamse Wiskunde Olympiade 2022 (ed.37): Azteekse diamant met 10 treden, in limiet naar $\infty$. 
\end{itemize}

Mijn zienswijze op definitie Cauchyrij: de definitie gaat eigenlijk over een soort tunnel (op een $\mathbb{N}-\mathbb{R}$ ) op het einde van de rij, die willekeurig smal kan gekozen worden. Want elke 2 willekeurige punten liggen nooit verder uit mekaar dan een willekeurig kleine afstand. Die tunnel begint op een gekozen waarde voor $n$, maar loopt dan wel door tot in het oneindige. Zo'n oneindig lange, en willekeurig smalle tunnel, je voelt aan dat dit equivalent is met convergentie. 


\section{KEEP ME TO JUMP TO END}

\end{document}