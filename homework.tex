% to change the appearance of the header, questions, problems or subproblems, see the homework.cls file or
% override the \Problem, \Subproblem, \question or \printtitle commands.

% The hidequestions option hides the questions. Remove it to print the questions in the text.
\documentclass[hidequestions]{homework}

\usepackage{amsthm}
\renewcommand*{\proofname}{Bewijs}

% Set up your name, the course name and the homework set number.
\homeworksetup{
    username={Raf Meeusen},
    course={Bewijzen en redeneren},
    setnumber=2}
\begin{document}% this also prints the header.

% use starred problems or subproblems to apply manual numbering.
\problem*{1}


Gegeven:
\[a, b, c, d \in \mathbb{R}\] 
\begin{equation}\label{gegeven1}a + b = c + d\end{equation}
\begin{equation}\label{gegeven2}ab = cd\end{equation}

Te bewijzen: 
ofwel geldt: \(a=c\) en \(b=d\) , ofwel geldt: \(a=d\) en \(b=c\). 


\begin{proof}
Laat ons eerst het 'te bewijzen' nader beschouwen. 
Stel dat \(a=c\), dan volgt uit (\ref{gegeven1}) dat \(b=d\). Op deze manier kunnen we inzien dat zodra we \'{e}\'{e}n van de volgende gelijkheden hebben aangetoond, het vervolg van het bewijs triviaal is: 

\(a=c\) of \(a=d\) of \(b=c\) of \(b=d\). 


Als \(a = 0\), dan volgt uit (\ref{gegeven2}) dat \(c = 0\) of  \(d = 0\), waaruit volgt dat \(a = c\) of \(a = d\), en de stelling bewezen is. Een gelijkaardige redenering kan gevolgd worden voor de gevallen \(b = 0\), \(c = 0\) en \(d = 0\). 

Het vervolg van het bewijs gaat over de gevallen waarbij  \(a \neq 0\),  \(b \neq 0\),  \(c \neq 0\) en  \(d \neq 0\). Beschouwen we de vergelijkingen (\ref{gegeven1}) en (\ref{gegeven2}) als een stelsel van vergelijkingen in de veranderlijken \(a\) en \(c\), met \(b\) en \(d\) constanten, en herschrijven we dit stelsel als: 
\[\left\{ 
\begin{aligned}
a-c &= d-b\\
ba -dc &= 0 
\end{aligned} 
\right.\]

De uitbreide matrix voor dit stelsel is dan: 
$\begin{pmatrix}
1 & -1 & d-b\\
b & -d & 0
\end{pmatrix}$

Aangezien \(b \neq 0\) mogen we de rijoperatie $R_1 \longrightarrow R_1 - \frac{1}{b}R_2$ doen en krijgen we: 
\begin{equation}\label{matrix1}
\begin{pmatrix}
0 & \frac{d-b}{b} & d-b\\
b & -d & 0
\end{pmatrix}
\end{equation}

Als \(b=d\), dan is de stelling sowieso bewezen. Als \(b \neq d\), dan heeft het stelsel met matrix (\ref{matrix1}) exact 1 oplossing, waarbij uit de eerste rij volgt dat \(c=b\), en de stelling ook bewezen is. 

\end{proof}





\problem

Neem aan dat de getallen \[a_0, a_1, a_2, . . .\] voldoen aan \[a_0 = a_1 = 1\] en

\[a_{n+1} a_{n-1} = pa_n^2\] 

voor \[n \in \mathbb{N}_0\]
 
Hierin is \[p > 0\] een vast gekozen re\"{e}el getal. 
Bewijs dat \[a_n = p\] 



\end{document}


